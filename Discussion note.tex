\documentclass{article}

% Language setting
% Replace `english' with e.g. `spanish' to change the document language
\usepackage[english]{babel}

% Set page size and margins
% Replace `letterpaper' with`a4paper' for UK/EU standard size
\usepackage[letterpaper,top=2cm,bottom=2cm,left=3cm,right=3cm,marginparwidth=1.75cm]{geometry}

% Useful packages
\usepackage{amsmath}
\usepackage{amsfonts}
\usepackage{amsthm}
\allowdisplaybreaks
\numberwithin{equation}{section}
\newtheorem{theorem}{Theorem}[section]
\newtheorem{corollary}{Corollary}[theorem]
\newtheorem{lemma}{Lemma}[section]
\newtheorem{assumption}{Assumption}[section]
\newtheorem{definition}{Definition}[section]

\usepackage{algorithm}
\usepackage{algorithmic}

\usepackage{graphicx}
\usepackage{float} 
\usepackage{subfigure}
\usepackage[colorlinks=true, allcolors=blue]{hyperref}
\bibliographystyle{plain}
\allowdisplaybreaks[2]

\def\Argmin{\mathop{\mathrm{Argmin}}}


\title{Discussion Note}
\author{Zhang Hao}

\begin{document}
\maketitle


\section{Introduction }
Problems whose structure involves a separable objective function and variables subject to linear constraints are
common in the fields of statistics, machine learning, scientific computing, etc. Many constrained and unconstrained
optimization problems can be reformulated in this form. 

A classic method to solve these method is ADMM. However, in each iteration, ADMM needs to compute the projection, 
which can be time-consuming and memory-intensive. Therefore, in order to overcome this drawback, we can use other first-order
algorithms such as Frank-Wolfe method.

The Frank-Wolfe type method, also known as conditional gradient method, has become quite popular recently because it only requires to minimize the LO which is the linear approximation of the
objective function. Compared to computing the projection in proximal gradient method, solving LO can be much more
efficient in practice.

In this paper, we consider the following problem
\begin{align} \label{problem}
    &\min f(x)+g(y) \nonumber  \\
    &s.t.\hspace{2pt} Ax+By=c
\end{align}
where $A: \mathcal{E}_1 \rightarrow \mathcal{E}$, $B: \mathcal{E}_2 \rightarrow \mathcal{E}$ are linear maps, 
$f: \mathcal{E}_1 \rightarrow \mathbb{R}\cup\{\infty\}$, 
$g: \mathcal{E}_2 \rightarrow \mathbb{R}\cup\{\infty\}$ are two proper closed convex functions 
and may not be smooth. In some application scenarios, it is easy to compute the projection onto $\mathrm{dom} f$. However, the projection onto $\mathrm{dom} g$ is not as straightforward. In contrast, 
it is easy to solve the LO of $g$. Therefore, we can get $x_{k+1}$ by proximal gradient method and get $y_{k+1}$ by Frank-Wolfe
method.  The details of algorithm is in Section \ref{section_convergence_Lipschitz}.

In this paper, we first consider the situation that $f$ and $g$ have Lipschitz continous gradient. In this case, 
we find that the function value $f(x^k)+ g(y^k)$ converges to the optimal value of problem \eqref{problem}
at the rate of $\mathcal{O}\left( \frac{1}{\sqrt{k}} \right)$. Additionally, the iteration points $\{x^k\}$, $\{y^k\}$ 
satisfy the property that $\|Ax^k+By^k-c\rVert$  converges to 0 at the rate of $\mathcal{O}\left(\frac{1}{\sqrt{k}}\right)$ 
under some moderate assumptions. 
Furthermore, we study the convergence rate of the sequence $\{x^k,y^k\}$ based on the 
Kurdyka-\L{}ojasiewicz(KL) property. Finally, we present the numerical results of Hankel matrix optimization
problem which has the form of \eqref{problem}.  

Throughout this paper, let $D_x$ and $D_y$ denote the diameters of $\mathrm{dom}$ $f$ 
and $\mathrm{dom}$ $g$ respectively, i.e.,
\begin{align} \label{Definition_of_diameter}
    D_x = \sup\limits_{x_1,x_2\in \mathrm{dom}f}\|x_1 -x_2\rVert,
    D_y = \sup\limits_{y_1,y_2\in \mathrm{dom}g}\|y_1 -y_2\rVert
\end{align}

\begin{assumption} \label{Assumption 1}
    Assume that $f = f_1 +f_2$, 
    where $f_1$ is convex and smooth while $f_2$ is nonsmooth.
    Besides, assume that   
    $g = g_1 +g_2$, where $g_1$ is convex and smooth while $g_2$ is nonsmooth. 
\end{assumption}

\begin{assumption} \label{Assumption 2}
    Assume that $\mathrm{dom} f$ and $\mathrm{dom} g$ are bounded, i.e, $D_x<\infty$ and $D_y < \infty$. 
\end{assumption}

\section{Notation and preliminaries}
Throughout this paper, we use $\mathbb{N}$ to denote the set of nonnegative integers and $\mathcal{E}$, $\mathcal{E}_1$, $\mathcal{E}_2$ to denote the 
finite dimensional real Hilbert spaces. For two vectors $x$ and $y$ in $\mathbb{R}^n$, we use $\langle x,y\rangle$
to denote the inner product and $\|x\rVert$ to denote the Euclidean norm. Besides, for two matrices 
$X$ and $Y$ in $\mathbb{R}^{m\times n}$, the Frobenius norm is denoted by $\|X\rVert_F$ and 
the inner product is denoted by $\langle X,Y \rangle =\mathrm{tr}\left(X^TY\right)$.
Let $A: \mathcal{E}_1\rightarrow \mathcal{E}$ be a linear operator. We use $\lambda_{\max}(A^*A)$ to denote the maximum eignvalue value of $A^TA  $.

For a proper closed convex function $f$, we use $\partial f(x)$ to denote the subdifferential of $f$ at $x$, 
i.e.
\begin{align}
    \partial f(x) = \left\{\xi \in \mathbb{R}^n: f(y)- f(x) \geq\langle \xi, y-x \rangle \hspace{4pt} \forall y \in \mathbb{R}^n\right\}
    \nonumber
\end{align}
Besides, for a nonempty closed convex set $\mathcal{C}$, we use $\delta_{\mathcal{C}}$ to denote the indicator function, which
is defined as 
\begin{align}
    \delta_{\mathcal{C}}(x) = \left\{ 
        \begin{array}{cc}
            0  & x\in \mathcal{C},  \nonumber \\
            \infty & x \notin \mathcal{C}.  \nonumber
        \end{array}
     \right.  \nonumber
\end{align}
The normal cone of $\mathcal{C}$ at an $x\in\mathcal{C}$ is 
defined as $\mathcal{N}_{\mathcal{C}}(x):=\partial\delta_{\mathcal{C}}(x)$. 

In order to analyze the convergence in Section \ref{section_convergence_Lipschitz}, we also introduce the Abel's summation formula. Let
$\{a_k\}$ and $\{b_k\}$ be two sequences. Then, 
\begin{align}
    \sum\limits_{k=0}^n a_k(b_k - b_{k+1}) = (a_0b_0 - a_{n}b_{n+1})+ \sum\limits_{k=1}^{n} b_{k}(a_{k}-a_{k-1})
    \label{Abel formula}
\end{align}

Next, we introduce some important definitions that will be used in our convergence analysis. To start with, 
we consider the following contraint qualification: 
\begin{definition}
    {\rm{\textbf{(CQ).}}} Recall that $f_2$ and $g_2$ are defined in Assumption \ref{Assumption 1}. 
    We say the \textbf{CQ} holds if the following statement holds. 
    \begin{align}
        c \in A\hspace{2pt} \mathrm{ri}\left(\mathrm{dom}\hspace{2pt}f_2\right) +
    B\hspace{2pt} \mathrm{ri}\left(\mathrm{dom}\hspace{2pt}g_2\right) \label{CQ} 
    \end{align}
\end{definition}
\begin{lemma}
    {\rm{\textbf{(Optimality condition).}}} \label{optimal}
    Suppose that the constraint qualification \eqref{CQ} holds. 
    We say that $x^*\in \mathrm{dom}f$ and $y^*\in \mathrm{dom}g$ are minimizers of problem $\eqref{problem}$ 
    if and only if there exists a vector $\bar{\lambda} \in \mathcal{E}$ such that the following statements hold.
    
    $(\romannumeral1)$ $0 \in \partial f(x^*) + A^*\bar{\lambda}  $

    $(\romannumeral2)$ $0 \in \partial g(y^*) + B^*\bar{\lambda} $

    $(\romannumeral3)$ $Ax^* + By^* =c$
    
\end{lemma}

Finally, we introduce the definitions and some useful lemmas of Kurdyka-\L{}ojasiewicz property and exponent.
\begin{definition}
    {\rm{\textbf{(KL property and exponent).}}} Suppose that $f$ is a proper closed convex function and it satisfies KL property 
    at an $\bar{x}\in \mathrm{dom} \partial f$ if there exist $r\in (0,\infty]$, a neighborhood $U$ of $\bar{x}$
    and a continuous concave function $\phi: [0,r) \rightarrow \mathbb{R}_+$ such that

    {\rm{(\romannumeral1)}} $\phi(0) = 0$, $\phi$ is differentiable on $(0,r)$ and $\phi' >0$.

    {\rm{(\romannumeral2)}} $\forall x\in U$ and $f(\bar{x}) \leq f(x) < f(\bar{x}) +r$, it holds that
                            \begin{align}
                                \phi'(f(x)-f(\bar{x}))\mathrm{dist}(0,\partial f(x)) \geq 1 \nonumber
                            \end{align}
    furthermore, if $\phi(x) = \rho x^{1-\alpha}$ for some $\rho>0$ and $\alpha\in[0,1)$. Then we can 
    say that $f$ satisfies the KL property with exponent $\alpha$ at $\bar{x}$. 
\end{definition}

The next lemma taken from Lemma 3.10 in \cite{bolte2014proximal} plays an important role in analyzing the convergence rate of the whole sequence
$\{x^k,y^k\}$. 
\begin{lemma} \label{error_bound}
    Suppose that $f$ is a proper, closed, convex and level-bounded function. 
    Let $\Lambda := \Argmin f \neq \emptyset$. Suppose that $f$ satisfies the KL property with exponent $\alpha$ 
    at each point in $\Lambda$. Then there exist $\epsilon>0$, $r_0>0$ and $c_0>0$ such that
    \begin{align}
        \mathrm{dist}(x,\Lambda) \leq c_0 (f(x)-\inf f)^{1-\alpha} \nonumber
    \end{align}
    for any $x\in \partial f$ with $dist(x,\Lambda) \leq \epsilon$ and $\inf f <f(x) < \inf f+ r_0$.
\end{lemma}

\section{Algorithm framework under Lipschitz continous assumption}\label{section_convergence_Lipschitz} 
Before introducing our method for solving problem $\eqref{problem}$ under Assumption \ref{Assumption 1}, recall that $f=f_1 + f_2$, 
$g=g_1+g_2$ and  $f_1,g_1$ are smooth. 
Besides, our method for solving problem \eqref{problem} also relies on the following assumption. 
\begin{assumption} \label{Assumption_Lipschitz_continuity}
    Suppose that $f_1$ is smooth and has Lipschitz continuous gradient with constant $L_f$. Similarly, 
    we can also suppose that $g_1$ is smooth and has Lipschitz continuous gradient with constant $L_g$.
\end{assumption}

Our algorithm is inspired by \cite{yu2020rc}. First, we use quadratic 
penalty function to modify  problem $\eqref{problem}$ which is shown in $\eqref{problem_2}$ 
with penalty parameter $\beta_k$:
\begin{align}
    \min\limits_{x\in \mathcal{E}_1,y \in \mathcal{E}_2} f(x) + g(y) + \beta_k h(x,y) 
    \label{problem_2}
\end{align}
where $h(x,y) = \frac{1}{2}\|Ax+ By-c \rVert^2$.

Next, compared to ADMM algorithm, we use proximal gradient method to obtain $x^{k+1}$ while use Frank
Wolfe method to obtain $y^{k+1}$. Finally, we update $\beta_k$ by a certain rule. In Frank Wolfe step, we use the diminishing 
stepsize $\alpha_k = \frac{2}{k+2}$. Notice that the penalty function 
is quadratic. Therefore, to make sure that the algorithm is convergent, $\beta_k$ should be infinite 
as $k\rightarrow \infty$. The full algorithm for solving problem \eqref{problem} is shown in Algorithm \ref{alg1}.


\begin{algorithm}
	%\textsl{}\setstretch{1.8}
	\caption{Basic Algorithm}
	\label{alg1}
	\begin{algorithmic}
		\STATE \textbf{Step 0} Choose $x^0 \in \mathrm{dom} f, y^0 \in \mathrm{dom} g$, $\beta_0>0$ and the positive sequences $\{\alpha_t\}$, $\{\beta_t\}$ 
        with $\lim\limits_{t\rightarrow\infty}\alpha_t=0$ and $\lim\limits_{t\rightarrow \infty}\beta_t=\infty $. 
        Let $\lambda_{A}=\lambda_{\max}(A^*A)$
		\STATE \textbf{Step 1} For $t=0,1,\cdots$ compute   
            \begin{align}
                &x^{t+1} \in \Argmin \langle \nabla f_1(t^k) + \beta_{t}A^*(Ax^t+By^t-c), x-x^t\rangle +\frac{L_f+\lambda_A \beta_t}{2}\|x-x^t\rVert^2 +f_2(x) \label{get_x_k} \\
                &u^t \in \Argmin \langle\nabla g_1(y^t) + \beta_{t}B^*(Ax^{t+1} +By^t-c),y\rangle +g_2(y) \label{get_uk} \\
                &y^{t+1} = y^{t} +\alpha_t(u^t-y^t)  \label{get_y_k}
            \end{align}
	\end{algorithmic}  
\end{algorithm}

In order to establish the convergence results of the Algorithm \ref{alg1}, we first present some auxiliary lemmas.
\begin{lemma} \label{lemma_of_rk}
    Let $\delta \in (0,1]$  and suppose that $\beta_{t} = (t+1)^{\delta }$ and $\alpha_t = \frac{2}{t+2}$ for all $t\in \mathbb{N}$. Then, we have that 
    \begin{align}
        (1-\alpha_t)(\beta_{t}-\beta_{t-1}) \leq \alpha_t \beta_{t}
    \end{align} 
\end{lemma}
\begin{proof}
    \begin{align}
        &(1-\alpha_t)(\beta_{t}-\beta_{t-1}) -\alpha_t \beta_{t} = \frac{t}{t+2}\left((t+1)^{\delta}
        -t^{\delta}\right) -\frac{2}{t+2}(t+1)^{\delta} \nonumber \\
        & \overset{\mathop{(a)}}{\leq} \frac{t}{t+2}\delta t^{\delta -1} -\frac{2}{t+2}(t+1)^{\delta} 
        \overset{\mathop{(b)}}{\leq} \frac{t^{\delta}}{t+2} -\frac{2(t+1)^{\delta}}{t+2} \leq -\frac{(t+1)^{\delta}}{t+2} \leq 0
        \nonumber
    \end{align}
    where $(a)$ holds because the function $x \mapsto x^{\delta}$ is concave and $(b)$ holds
    because $\delta \leq 1$.
\end{proof}

\begin{lemma} \label{lemma_of_sum_Axk-Axk+1}
    Let  $\{x^t\}$, $\{y^t\}$ be generated in Algorithm \ref{alg1}, $y^*$ be a point 
    in $\mathrm{dom}$g, $\alpha_t = \frac{2}{t+2}$ and $\beta_{t} =(t+1)^{\delta}$ for some $\delta \in (0,1]$. 
    Then we have for each $k\in\mathbb{N}$ that  
    \begin{align}
        &\left| \sum\limits_{t = 0}^k\alpha_t\beta_{t}(t+1)(t+2)\left\langle Ax^t- Ax^{t+1},By^t-By^* \right\rangle \right| \nonumber \\
        \leq& 2 \beta_0 \left|\left\langle Ax^0, By^* \right\rangle \right|
        + 2(k+1)^{1+\delta} \left|\left\langle  Ax^{k+1},By^* \right\rangle\right|
        +2 \beta_0 \left| \left\langle Ax^0, By^0 \right\rangle \right|
        + 2(k+1)^{1+\delta} \left| \left\langle  Ax^{k+1},By^k \right\rangle\right| \nonumber \\
        &+ 2(1+\delta) \sum\limits_{t=1}^{k-1} (t+2)^{\delta} \left|\left\langle Ax^t, By^* \right\rangle\right|
        + 2(1+\delta) \sum\limits_{t=1}^{k-1} (t+2)^{\delta} \left| \left\langle Ax^t, By^{t+1} \right\rangle\right| \nonumber \\
        &+ 4 \sum\limits_{t=1}^{k-1}  (t+1)^{\delta} \left| \left\langle Ax^t, Bu^t-By^t \right\rangle \right|
    \end{align}
\end{lemma}
\begin{proof}
    First, we have that 
    \begin{align}
        &\left| \sum\limits_{t = 0}^k\alpha_t\beta_{t}(t+1)(t+2)\left\langle Ax^t- Ax^{t+1},By^t-y^* \right\rangle \right| \nonumber \\
        &\leq \left| \sum\limits_{t = 0}^k\alpha_t\beta_{t}(t+1)(t+2)\left\langle Ax^t- Ax^{t+1},By^* \right\rangle \right|
        + \left| \sum\limits_{t = 0}^k\alpha_t\beta_{t}(t+1)(t+2)\left\langle Ax^t- Ax^{t+1},By^t \right\rangle \right|
        \label{sum_Axk-Axk+1_two_parts}
    \end{align}
    Next, in order to further estimate the first term in \eqref{sum_Axk-Axk+1_two_parts}, 
    we use the Abel's summation formula by letting $a_k = \alpha_kr_k(k+1)(k+2)$ and $b_k = \langle Ax^k,By^* \rangle$ in \eqref{Abel formula}.
    \begin{align} 
        &\left| \sum\limits_{t = 0}^k\alpha_t\beta_{t}(t+1)(t+2)\left\langle Ax^t- Ax^{t+1},By^* \right\rangle \right| \nonumber \\
        \overset{\mathop{(a)}}{=}& \Bigg| 2\alpha_0\beta_0 \left\langle Ax^0, By^* \right\rangle -\alpha_k\beta_{k}(k+1)(k+2)\left\langle  Ax^{k+1},By^* \right\rangle \nonumber \\
        &+ \left. \sum\limits_{t=1}^{k-1} \left[ -\alpha_t\beta_{t}(t+1)(t+2)\left\langle Ax^t, By^* \right\rangle
        +  \alpha_{t+1}\beta_{t+1}(t+2)(t+3)\left\langle Ax^t, By^* \right\rangle\right] \right|    \nonumber \\
        \leq &\left| 2\alpha_0\beta_0 \left\langle Ax^0, By^* \right\rangle \right| +\left|\alpha_k\beta_{k}(k+1)(k+2)\left\langle  Ax^{k+1},By^* \right\rangle\right| \nonumber \\
        &+\left| \sum\limits_{t=1}^{k-1} \left(\alpha_{t+1}\beta_{t+1}(t+2)(t+3) -\alpha_t\beta_{t}(t+1)(t+2)\right)\left\langle Ax^t, By^* \right\rangle\right| \nonumber \\
        \overset{\mathop{(b)}}{\leq} &2\left| \beta_0 \left\langle Ax^0, By^* \right\rangle \right|
        + 2\left|(k+1)^{1+\delta}\left\langle  Ax^{k+1},By^* \right\rangle\right|
        +2\left| \sum\limits_{t=1}^{k-1} \left((t+2)^{1+\delta} -(t+1)^{1+\delta}\right)\left\langle Ax^t, By^* \right\rangle\right| 
        \nonumber \\ 
        \overset{\mathop{(c)}}{\leq} & 2 \beta_0 \left|\left\langle Ax^0, By^* \right\rangle \right|
        + 2(k+1)^{1+\delta}\left|\left\langle  Ax^{k+1},By^* \right\rangle\right|
        +2(1+\delta) \sum\limits_{t=1}^{k-1} (t+2)^{\delta} \left|\left\langle Ax^t, By^* \right\rangle\right| 
        \label{Axk_Axk+1,By*}
    \end{align}
    where $(a)$ holds due to Abel's summation formula, $(b)$ holds due to the definitions 
    of $\alpha_t$, $\beta_{t}$ and $\alpha_t\beta_t(t+1)(t+2)=2(t+1)^{1+\delta}$ and $(c)$ holds because
    the function $x \mapsto x^{1+\delta}$ is convex.
    
    To further analyze the second term in \eqref{sum_Axk-Axk+1_two_parts}, we use the method which is 
    similar to Abel' summation formula. 
    \begin{align}
        &\left| \sum\limits_{t = 0}^k\alpha_t\beta_{t}(t+1)(t+2)\left\langle Ax^t- Ax^{t+1},By^t \right\rangle \right| \nonumber \\
        \overset{(a)}{=}& \Bigg| 2\alpha_0\beta_0 \langle Ax^0, By^0 \rangle -\alpha_k\beta_k(k+1)(k+2)\langle Ax^{t+1}, By^t \rangle  \nonumber \\
        &+\left.\sum\limits_{t=0}^k\left(-\alpha_t\beta_t(t+1)(t+2)\langle Ax^t, By^t \rangle+\alpha_{t+1}\beta_{t+1}(t+2)(t+3)
        \langle Ax^{t}, By^{t+1} \rangle\right) \right| \nonumber \\
        =& \Bigg| 2\alpha_0\beta_0 \langle Ax^0, By^0 \rangle -\alpha_k\beta_k(k+1)(k+2)\langle Ax^{t+1}, By^t \rangle  \nonumber \\
        &+\sum\limits_{t=0}^n\left(-\alpha_t\beta_t(t+1)(t+2)\langle Ax^t, By^t \rangle+\alpha_t\beta_{t}(t+1)(t+2) 
        \langle Ax^{t}, By^{t+1} \rangle\right)  \nonumber \\
        & \left.+ \sum\limits_{t=0}^k\left(\alpha_{t+1}\beta_{t+1}(t+2)(t+3) - \alpha_t\beta_t(t+1)(t+2)\langle Ax^t, By^{t+1} \rangle\right) \right| \nonumber \\
       \leq & \left| 2\beta_0 \left\langle Ax^0, By^0 \right\rangle \right|
        + \left|2(k+1)^{1+\delta}\left\langle  Ax^{k+1},By^k \right\rangle\right| +\left| \sum\limits_{t=1}^{k-1}\alpha_t \beta_{t}(t+1)(t+2) \left\langle Ax^t, By^{t+1}-By^t \right\rangle \right| \nonumber \\
        & + \left|\sum\limits_{t=1}^{k-1}\left(\alpha_{t+1}\beta_{t+1}(t+2)(t+3) - \alpha_t\beta_{t}(t+1)(t+2)\right)\left\langle Ax^t, By^{t+1} \right\rangle \right| \nonumber \\
        \overset{\mathop{(b)}}{\leq} & 2 \beta_0 \left| \left\langle Ax^0, By^0 \right\rangle \right|
        + 2(k+1)^{1+\delta} \left| \left\langle  Ax^{k+1},By^k \right\rangle\right| 
        + \sum\limits_{t=1}^{k-1} \left| \alpha_t \beta_{t} (t+1)(t+2) \left\langle Ax^t, \alpha_tB(u^t-y^t) \right\rangle \right| \nonumber \\
        & +\left|\sum\limits_{t=1}^{k-1}\left(\alpha_{t+1}\beta_{t+1}(t+2)(t+3) - \alpha_t\beta_{t}(t+1)(t+2)\right)\left\langle Ax^t, By^{t+1} \right\rangle \right|
        \nonumber \\
        \overset{\mathop{(c)}}{\leq} & 2 \beta_0 \left| \left\langle Ax^0, By^0 \right\rangle \right|
        + 2(k+1)^{1+\delta} \left| \left\langle  Ax^{k+1},By^k \right\rangle\right| 
        + \sum\limits_{t=1}^{k-1} \left| \alpha_t^2 \beta_{t} (t+1)(t+2) \left\langle Ax^t, B(u^t-y^t) \right\rangle \right| \nonumber \\
        & +2(1+\delta) \sum\limits_{t=1}^{k-1} (t+2)^{\delta} \left| \left\langle Ax^t, By^{t+1} \right\rangle\right|
        \nonumber \\
        \overset{\mathop{(d)}}{\leq} &2 \beta_0 \left|\left\langle Ax^0, By^0 \right\rangle \right|
        + 2(k+1)^{1+\delta}\left|\left\langle  Ax^{k+1},By^k \right\rangle\right| 
        + 4 \sum\limits_{t=1}^{k-1} \left|\beta_{t}\left\langle Ax^t, Bu^t-By^t \right\rangle \right| \nonumber \\
        & +2(1+\delta) \sum\limits_{t=1}^{k-1} (t+2)^{\delta} \left| \left\langle Ax^t, By^{t+1} \right\rangle\right|
        \label{Axk_Axk+1,Byk}
    \end{align}
    where $(a)$ holds by rearranging the terms which is similar to Abel's formula, $(b)$ holds due to \eqref{get_y_k}, $(c)$ holds 
    because the function $x \mapsto x^{1+\delta}$ is convex and $\alpha_t\beta_t(t+1)(t+2) = 2(t+1)^{1+\delta}$. 
    Besides, $(d)$ holds because of the definition of $\beta_t$ and $\alpha_t^2(t+1)(t+2)\leq 4$. 
    Combining  $\eqref{Axk_Axk+1,By*}$ with $\eqref{Axk_Axk+1,Byk}$, we obtain the desired result.
\end{proof}


\section{Convergence analysis}
\subsection{Convergence rate analysis}
\begin{theorem} \label{theorem_convergence_Lipschitz}
    Consider problem \eqref{problem} and the Algorithm \ref{alg1} for solving problem \eqref{problem}. 
    Suppose that the Assumptions \ref{Assumption 1}, \ref{Assumption 2}, \ref{Assumption_Lipschitz_continuity} 
    and CQ \eqref{CQ} hold. Let $x^*$, $y^*$, and $\bar{\lambda}$ be defined in lemma \ref{optimal}, 
    $\alpha_k=\frac{2}{k+2}$ and $\beta_{k} =\sqrt{k+1}$ for all $k\in \mathbb{N}$. 
    Let the sequences $\{x^k\}$, $\{y^k\}$ and $\{u^k\}$ be generated by Algorithm \ref{alg1}. 
    Then it holds that for all $k\in \mathbb{N}$:
    \begin{align}
        \left|f(x^{k}) + g(y^{k})  - f(x^*) -g(y^*) \right| 
        \leq \max\left\{ \tau_k, \|\bar{\lambda} \rVert \gamma_k \right\}
    \end{align}
    and 
    \begin{align}
        \| Ax^k +By^k-c \rVert \leq \gamma_k
    \end{align}
    where
    \begin{align}
        &\tau_k=\frac{2c_1+2c_4}{\sqrt{k+1}} +\frac{c_2}{k+1}+ \frac{c_3}{k(k+1)} \nonumber \\
        &\gamma_k =\frac{\|\bar{\lambda}\rVert}{\sqrt{k}}+ \sqrt{\frac{\|\bar{\lambda}\rVert^2}{k}+\frac{4c_1+4c_4}{\sqrt{k(k+1)}} 
        +\frac{2c_2}{(k+1)\sqrt{k}}+ \frac{2c_3}{k\sqrt{k}(k+1)}} \nonumber 
    \end{align}
    and
    \begin{align}
        c_1= (2\lambda_{A} D_x^2+ 2\lambda_{B} D_y^2), c_2=2L_fD_x^2+2L_gD_y^2, c_3=2|\langle Ax^0,By^*\rangle|+2\left|\langle Ax^0, By^0\rangle \right|, c_4 = \frac{40}{3}D_2;
        \nonumber
    \end{align}
    here, $L_f$, $L_g$ are the Lipschitz constants of $f_1$, $g_1$ which are defined in Assumption \ref{Assumption_Lipschitz_continuity}, 
    $D_x$ and $D_y$ are given in \eqref{Definition_of_diameter}, $D_2 = \max\limits_{x\in\mathrm{dom}f,y\in\mathrm{dom}g}\left| \langle Ax,By\rangle \right|$,   
    $\lambda_{A} =\lambda_{\max}(A^*A)$, $\lambda_{B} = \lambda_{\max}(B^*B)$ and $h(x,y)$ is defined in \eqref{problem_2}. 
\end{theorem}
\begin{proof}
    Recall that $\lambda_{A} = \lambda_{\max}(A^*A)$, $h(x,y) =\frac{1}{2}\|Ax+By-c\rVert^2$.  
    Let $(x^t,y^t)$ be the $t$-th iteration result. Due to the convexity of $\mathrm{dom}f$, 
    we have that the point $x^t + \alpha_t(x^*-x^t)$ is also in $\mathrm{dom}f$.   
    Therefore, 
    \begin{align}
        &f_1(x^{t+1})+ \beta_{t}h(x^{t+1},y^k) +f_2(x^{t+1})  \nonumber \\
        \overset{\mathop{(a)}}{\leq}& f_1(x^t) +\beta_{t}h(x^t,y^t) +\left\langle \nabla f_1(x^t)+\beta_{t}A^*(Ax^t+By^t-c), x^{t+1} -x^t\right\rangle   \nonumber \\
        & + \frac{L_f+\lambda_{A} \beta_{t}}{2}\|x^{t+1} - x^{t}\rVert^2 +f_2(x^{t+1}) \nonumber \\
        \overset{\mathop{(b)}}{\leq}& f_1(x^t) +\beta_{t}h(x^t,y^t) +\left\langle \nabla f_1(x^t)+\beta_{t}A^*(Ax^t+By^t-c), x^t+\alpha_t(x^*-x^t) -x^t\right\rangle \nonumber \\
        &+ \frac{L_f+\lambda_{A} \beta_{t}}{2}\|x^t+\alpha_t(x^*-x^t) -x^t\rVert^2 +f_2\left(x^t+\alpha_t(x^*-x^t) \right)  \nonumber \\
        \overset{\mathop{(c)}}{\leq}& f_1(x^t) +\beta_{t}h(x^t,y^t) +\alpha_t\beta_{t}\left\langle  A^*(Ax^t+By^t-c), x^* -x^t\right\rangle+ \alpha_t\left( f_1(x^*)-f_1(x^t)\right)  \nonumber\\
        &+ \frac{L_f+\lambda_{A} \beta_{t}}{2}\alpha_t^2\|x^* - x^{t}\rVert^2 + \alpha_t f_2(x^*) +(1-\alpha_t)f_2(x^t)  \nonumber \\
        \overset{\mathop{(d)}}{=}& (1-\alpha_t)f(x^t) +\beta_{t}h(x^t,y^t) +\alpha_t f(x^*)+\alpha_t\beta_{t}\left\langle  A^*(Ax^t+By^t-c), x^* -x^t\right\rangle \nonumber \\
        &+ \frac{L_f+\lambda_{A} \beta_{t}}{2}\alpha_t^2\|x^* - x^{t}\rVert^2
        \label{f_1_poposition_1}
    \end{align}
    where $(a)$ holds because $f_1,\beta_th(\cdot,y^t)$ have Lipschitz gradients with constant $L_f$, $\beta_t, \lambda_{A}$ respectively, $(b)$ holds because $x^t+\alpha_t(x^*-x^t)$ is in $\mathrm{dom}f$ and $x^{t+1}$ 
    is the minimizer (see \eqref{get_x_k}), $(c)$ holds because $f_1$ and $f_2$ are convex, 
    $(d)$ holds because $f=f_1+f_2$. 

    Next, by rearranging the terms, we obtain that 
    \begin{align}
         &f(x^{t+1})+ \beta_{t}h(x^{t+1},y^k) -f(x^*) \nonumber \\
         \leq& (1-\alpha_t)\left(f(x^t) -f(x^*)\right) +\beta_{t}h(x^t,y^t)+\alpha_t\beta_{t}\langle Ax^t+By^t-c,A(x^*-x^t)\rangle \nonumber\\
         &+\frac{L_f+\lambda_{A} \beta_{t}}{2}\alpha_t^2\|x^* - x^{t}\rVert^2 \nonumber \\
         \overset{\mathop{(a)}}{\leq}& (1-\alpha_t)\left(f(x^t) -f(x^*)\right) +\beta_{t}h(x^t,y^t)+\alpha_t\beta_{t}\langle Ax^t+By^t-c,A(x^*-x^t)\rangle +\frac{L_f+\lambda_{A} \beta_{t}}{2}\alpha_t^2D_x^2 \label{f_k+1}
    \end{align}
    where $(a)$ holds because of the definition of $D_x$ in \eqref{Definition_of_diameter}.

    Next, recalling that $\lambda_{B} = \lambda_{\max}(B^*B)$, we have that 
    \begin{align}
        &g(y^{t+1}) +\beta_{t}h(x^{t+1},y^{t+1}) -g(y^*) \nonumber \\
        \overset{(a)}{\leq} &g_1(y^t) +\beta_{t}h(x^{t+1},y^t)+\langle \nabla g_1(y^t)+\beta_{t}B^*(Ax^{t+1}+By^t-c), y^{t+1} -y^t \rangle \nonumber\\
        &+\frac{L_g +\lambda_{B} \beta_{t}}{2}\|y^{t+1} -y^t\rVert^2 +g_2(y^{t+1}) -g(y^*) \nonumber \\
        \overset{\mathop{(b)}}{\leq}& g_1(y^t)+\beta_{t}h(x^{t+1},y^t) +\alpha_t\langle \nabla g_1(y^t)+\beta_{t}B^*(Ax^{t+1}+By^t-c), u^t -y^t \rangle \nonumber\\
        &+\frac{L_g +\lambda_{B} \beta_{t}}{2}\alpha_t^2\|u^t -y^t\rVert^2 +g_2(y^{t+1}) -g(y^*)
        \label{g_yk-gy*_1}
    \end{align}
    where $(a)$ holds because $g_1$ and $\beta_{t}h(x^{t+1},\cdot)$ have Lipschitz gradients with 
    constants $L_g$ and $\beta_t\lambda_{B}$ respectively, $(b)$ holds because $y^{t+1} = y^t+\alpha_t(u^t-y^t)$ 
    (see \eqref{get_y_k}). 
    Notice that $g_2$ is convex and we get that 
    \begin{align}
        &\alpha_t\langle \nabla g_1(y^t)+\beta_{t}B^*(Ax^{t+1}+By^t-c), u^t -y^t \rangle  +g_2(y^{t+1}) \nonumber \\
        \leq & \alpha_t\langle \nabla g_1(y^t)+\beta_{t}B^*(Ax^{t+1}+By^t-c), u^t -y^t \rangle  +\alpha_tg_2(u^t) +(1-\alpha_t)g_2(y^t) \nonumber \\
        \overset{\mathop{(a)}}{\leq}& \alpha_t\langle \nabla g_1(y^t)+\beta_{t}B^*(Ax^{t+1}+By^t-c), y^* -y^t \rangle  +\alpha_tg_2(y^*) +(1-\alpha_t)g_2(y^t) \nonumber\\
        \leq& \alpha_t(g_1(y^*)-g_1(y^t)) +\alpha_t\beta_t \langle Ax^{t+1}+By^t-c, B(y^* -y^t) \rangle +\alpha_tg_2(y^*) +(1-\alpha_t)g_2(y^t)
        \label{g_yk-gy^*_2}
    \end{align}
    where $(a)$ holds because of $\eqref{get_uk}$ and the last inequality holds because $g_1$ is convex.
    Therefore, we have upon combining \eqref{g_yk-gy*_1} and \eqref{g_yk-gy^*_2} that
    \begin{align}
        &g(y^{t+1}) +\beta_{t}h(x^{t+1},y^{t+1}) -g(y^*) \nonumber \\
        \leq& (1-\alpha_t)(g(y^t)-g(y^*))+\beta_{t}h(x^{t+1},y^t)+ \alpha_t\beta_{t}\langle Ax^{t+1} +By^t -c, By^* -By^t\rangle \nonumber \\
        &+\frac{L_g+\lambda_{B} \beta_t}{2} \alpha_t^2 \|u^t-y^t\rVert^2 \nonumber \\
        \leq& (1-\alpha_t)(g(y^t)-g(y^*))+\beta_{t}h(x^{t+1},y^t)+ \alpha_t\beta_{t}\langle Ax^{t+1} +By^t -c, By^* -By^t\rangle +\frac{L_g+\lambda_{B} \beta_t}{2} \alpha_t^2D_y^2
        \label{g_k+1}
    \end{align}
    where the last inequality follows from the definition of $D_y$ in \eqref{Definition_of_diameter}.  
    Summing $\eqref{f_k+1}$ and $\eqref{g_k+1}$, we have that 
    \begin{align}
        &f(x^{t+1}) +g(y^{t+1}) +\beta_{t}h(x^{t+1},y^{t+1}) -f(x^*) -g(y^*) \nonumber \\
        \leq& (1-\alpha_t)\left(f(x^t) +g(y^t) -f(x^*)-g(y^*)\right) +\beta_{t}h(x^t,y^t)+\alpha_t\beta_{t}\langle Ax^t+By^t-c,Ax^*-Ax^t\rangle \nonumber \\
        &+ \alpha_t\beta_{t}\langle Ax^{t+1} +By^t -c, By^* -By^t\rangle +\frac{L_f+\lambda_{A} \beta_{t}}{2}\alpha_t^2D_x^2+ \frac{L_g +\lambda_{B} \beta_{t}}{2}\alpha_t^2D_y^2 \nonumber \\
        =& (1-\alpha_t)\left(f(x^t) +g(y^t)+\beta_{t-1}h(x^t,y^t) -f(x^*)-g(y^*)\right)+\alpha_t\beta_{t}\langle Ax^t+By^t-c,Ax^*-Ax^t\rangle \nonumber \\
        &+ \alpha_t\beta_{t}\langle Ax^{t+1} +By^t -c, By^* -By^t\rangle +\frac{L_f+\lambda_{A} \beta_{t}}{2}\alpha_t^2D_x^2+ \frac{L_g +\lambda_{B} \beta_{t}}{2}\alpha_t^2D_y^2
        \nonumber \\
        & +(1-\alpha_t)(\beta_{t}-\beta_{t-1})h(x^t,y^t) + \alpha_t\beta_{t} h(x^t,y^t) \label{F_k-F_k+1}
    \end{align}
    Next, we can notice that  
    \begin{align}
        &(1-\alpha_t)(\beta_{t}-\beta_{t-1})h(x^t,y^t) + \alpha_t\beta_{t}h(x^t,y^t) + \alpha_t\beta_{t}\langle Ax^t+By^t-c,Ax^*-Ax^t\rangle \nonumber \\
        &+ \alpha_t\beta_{t}\langle Ax^{t+1} +By^t -c, By^* -By^t\rangle \nonumber \\
        \overset{\mathop{(a)}}{\leq}& 2\alpha_t\beta_{t} h(x^t,y^t) + \alpha_t\beta_{t}\langle Ax^t+By^t-c,Ax^*-Ax^t\rangle+ \alpha_t\beta_{t}\langle Ax^{t+1} +By^t -c, By^* -By^t\rangle \nonumber \\
        \overset{\mathop{(b)}}{=}& \alpha_t\beta_t\|Ax^t+By^t-(Ax^*+ By^*)\rVert^2 +\alpha_t\beta_{t} \langle Ax^{t} +By^t-(Ax^*+By^*),A(x^*-x^t) \rangle \nonumber\\
        &+ \alpha_t\beta_{t}\langle Ax^{t+1} +By^t-(Ax^*+By^*), B(y^*-y^t) \rangle \nonumber \\
        =& {\alpha_t\beta_{t}}\|A(x^t-x^*)\rVert^2 + {\alpha_t\beta_{t}}\|B(y^t-y^*)\rVert^2 +2\alpha_t\beta_{t}\langle Ax^t-Ax^*, By^t-By^*\rangle -\alpha_t\beta_{t}\|A(x^t-x^*)\rVert^2       \nonumber \\
        & -\alpha_t\beta_{t}\|B(y^t-y^*)\rVert^2 +\alpha_t\beta_{t}\langle B(y^t-y^*), A(x^*-x^t) \rangle +\alpha_t\beta_{t}\langle A(x^{t+1}-x^*), B(y^*-y^t) \rangle  \nonumber \\
        =&\alpha_t\beta_t \langle Ax^t -Ax^{t+1}, By^t-By^* \rangle \label{F_k-F_k+1_2}
    \end{align} 
    where $(a)$ holds due to lemma \ref{lemma_of_rk}, $(b)$ holds because $Ax^*+ By^* =c$. 
    Combining  $\eqref{F_k-F_k+1}$ with $\eqref{F_k-F_k+1_2}$, we obtain that 
    \begin{align}
        &f(x^{t+1}) +g(y^{t+1}) +\beta_{t}h(x^{t+1},y^{t+1}) -f(x^*) -g(y^*) \nonumber \\
        \leq& (1-\alpha_t)\left(f(x^t) +g(y^t)+\beta_{t-1}h(x^t,y^t) -f(x^*)-g(y^*)\right) + \frac{L_f+\lambda_{A} \beta_{t}}{2}\alpha_t^2D_x^2+ \frac{L_g +\lambda_{B} \beta_{t}}{2}\alpha_t^2D_y^2 \nonumber \\
        &+\alpha_t\beta_{t}\langle Ax^t-Ax^{t+1}, By^t-By^*\rangle \label{f+g+rh}
    \end{align}
    Let $\Gamma_t =t(t+1) \left(f(x^t) +g(y^t)+\beta_{t-1}h(x^t,y^t) -f(x^*)-g(y^*)\right)$. 
    Then, multiplying $(t+1)(t+2)$ to the both sides of \eqref{f+g+rh} and invoking the fact that 
    $(t+1)(t+2)\alpha_t^2 = (t+1)(t+2)\frac{4}{(t+2)^2} \leq 4$, we have that 
    \begin{align}
        \Gamma_{t+1} \leq& \Gamma_t + \frac{L_f+\lambda_{A} \beta_{t}}{2}(t+1)(t+2)\alpha_t^2D_x^2+ \frac{L_g +\lambda_{B} \beta_{t}}{2}(t+1)(t+2)\alpha_t^2D_y^2 \nonumber \\
        & +\alpha_t\beta_{t} (t+1)(t+2)\langle Ax^t -Ax^{t+1}, By^t-By^{*} \rangle  \nonumber \\
        \leq& \Gamma_t + 2(L_f+\lambda_{A} \beta_{t})D_x^2+ 2(L_g +\lambda_{B} \beta_{t})D_y^2 +\alpha_t\beta_{t}(t+1)(t+2)\langle Ax^t- Ax^{t+1}, By^t-By^{*}\rangle \nonumber \\
        =& \Gamma_t + (2\lambda_{A} D_x^2+ 2\lambda_{B} D_y^2)\beta_{t} +2L_fD_x^2+2L_gD_y^2 + \alpha_t\beta_{t}(t+1)(t+2)\langle Ax^t- Ax^{t+1}, By^t-By^{*}\rangle 
    \end{align}
    Summing both sides of the above inequality from $t=0$ to $k-1$, we have that
    \begin{align}
        &\Gamma_k \leq \sum_{t = 0}^{k-1}\left[ (2\lambda_{A} D_x^2+ 2\lambda_{B} D_y^2) \beta_{t} + (2L_fD_x^2+2L_gD_y^2 ) \right] +\sum_{t=0}^{k-1}\alpha_t \beta_{t}(t+1)(t+2)\langle Ax^t-Ax^{t+1}, By^t -By^*\rangle\nonumber \\
        & \leq k\beta_{k-1}(2\lambda_{A} D_x^2+ 2\lambda_{B} D_y^2)+ k(2L_fD_x^2+2L_gD_y^2 ) +\sum_{t=0}^{k-1}\alpha_t \beta_{t}(t+1)(t+2)\langle Ax^t-Ax^{t+1}, By^t -By^*\rangle \label{iterate Gammak}
    \end{align}
    where the last inequality holds because $\beta_t \leq \beta_{k-1}$ for all $0 \leq t\leq k-1$.
    By the lemma \ref{lemma_of_sum_Axk-Axk+1} and recalling that $\beta_t = \sqrt{t+1}$, we obtain that
    \begin{align}
        &\sum_{t=0}^{k-1}\alpha_t \beta_t(t+1)(t+2)\langle Ax^t-Ax^{t+1}, By^t -By^*\rangle \leq 
        \left| \sum_{t=0}^{k-1} \alpha_t \beta_t(t+1)(t+2)\langle Ax^t-Ax^{t+1}, By^t -By^*\rangle \right| \nonumber \\
        \leq&  2\beta_0\left|  \left\langle Ax^0, By^* \right\rangle \right|
        + 2(k+1)^{\frac{3}{2}}\left|\left\langle  Ax^{k+1},By^* \right\rangle\right|
        +2\beta_0\left|  \left\langle Ax^0, By^0 \right\rangle \right|
        + 2(k+1)^{\frac{3}{2}}\left|\left\langle  Ax^{k+1},By^k \right\rangle\right| \nonumber \\
        &+ 3 \sum\limits_{t=1}^{k-2} \sqrt{t+2} \left|\left\langle Ax^t, By^* \right\rangle\right|
        + 3 \sum\limits_{t=1}^{k-2} \sqrt{t+2}\left|\left\langle Ax^t, By^{t+1} \right\rangle\right|
        +4 \sum\limits_{t=1}^{k-2}  \sqrt{t+1} \left| \left\langle Ax^t, Bu^t-By^t \right\rangle \right| \nonumber
    \end{align}
    Recall that Assumption \ref{Assumption 2} holds and hence  
    $D_2 = \max\limits_{x\in\mathrm{dom}f, y\in\mathrm{dom}g}\left| \langle Ax,By\rangle\right| < \infty$.
    Therefore, we obtain that 
    \begin{align}
        &\sum_{t=0}^{k-1} \alpha_t \beta_{t}(t+1)(t+2)\langle Ax^t-Ax^{t+1}, By^t -By^*\rangle \nonumber \\
        &\overset{\mathop{(a)}}{\leq} 4\beta_0D_2
        + 4(k+1)^{\frac{3}{2}}D_2+ 6D_2 \sum\limits_{t=1}^{k-1} \sqrt{t+2} 
        +8D_2 \sum\limits_{t=1}^{k-1}  \sqrt{t+1}  \nonumber \\
        & \overset{\mathop{(b)}}{\leq} 4\beta_0D_2
        + 4(k+1)^{\frac{3}{2}}D_2+ 6D_2 \int_{0}^{k+1} \sqrt{x} \hspace{1pt}dx 
        +8D_2 \int_0^{k} \sqrt{x} \hspace{1pt} dx \nonumber \\
        & = \left(4\beta_0+\frac{16}{3}k^{\frac{3}{2}}+ 8(k+1)^{\frac{3}{2}}\right) D_2 \leq \left(4\beta_0+\frac{40}{3}(k+1)^{\frac{3}{2}}\right)D_2
        \label{sum_Axk_Axk+1}
    \end{align}
    where $(a)$ holds because of the definition of $D_2$ and $\left| \langle Ax^t, Bu^t - By^t \rangle \right|
    \leq \left| \langle Ax^t, Bu^t \rangle \right| + \left|\langle Ax^t,By^t \rangle\right| \leq 2D_2$, 
    $(b)$ holds because the function $x \mapsto x^{\frac{3}{2}}$ is increasing. 
    Combining \eqref{iterate Gammak} with \eqref{sum_Axk_Axk+1}, we have that
    \begin{align}
        \Gamma_k \leq k^{\frac{3}{2}}c_1 + nc_2 + c_3 + c_4(k+1)^{\frac{3}{2}} \leq (c_1+c_4)(k+1)^{\frac{3}{2}} +nc_2 +c_3   
    \end{align}
    where $c_1= (2\lambda_{A} D_x^2+ 2\lambda_{B} D_y^2)$, $c_2=2L_fD_x^2+2L_gD_y^2$, $c_3=4\beta_0D_2$, $c_4 = \frac{40}{3}D_2$. 
    
    Recall that $\Gamma_k = k(k+1)\left(f(x^k)+ g(y^k)+ \beta_{k-1}h(x^k,y^k) -f(x^*)-g(y^*)\right)$, then we have that
    \begin{align}
        &f(x^k) +g(y^k)+\beta_{k-1}h(x^k,y^k) -f(x^*)-g(y^*) \leq \frac{(k+1)^{\frac{1}{2}}(c_1+c_4)}{k} + \frac{c_2}{k+1}+\frac{c_3}{k(k+1)}  \nonumber\\
        & \leq \frac{k+1}{k}\frac{c_1+ c_4}{\sqrt{k+1}} +\frac{c_2}{k+1} +\frac{c_3}{k(k+1)} 
        \leq \frac{2(c_1 + c_4)}{\sqrt{k+1}} +\frac{c_2}{k+1} +\frac{c_3}{k(k+1)} \label{Fk-F*_leq}
    \end{align}
    Besides, we also deduce from the above inequality that 
    \begin{align}
        f(x^k) +g(y^k) -f(x^*)-g(y^*) &\leq -\beta_{k-1}h(x^k,y^k)+
        \frac{2(c_1+ c_4)}{\sqrt{k+1}} +\frac{c_2}{k+1}+ \frac{c_3}{k(k+1)} \nonumber \\
        &\leq \frac{2c_1+2c_4}{\sqrt{k+1}} +\frac{c_2}{k+1}+ \frac{c_3}{k(k+1)} \label{Fk-F*_leq_one_size}
    \end{align} 
    where the last inequality holds because $\beta_{k-1}h(x^k,y^k) \geq 0$. 
    Next, recall that the optimal condition in Lemma \ref{optimal}. Assume that there exist $\xi_1 \in \partial f(x^*)$ and $\xi_2 \in \partial g(y^*)$, such that 
    \begin{align}
        -\xi_1 -A^*\bar{\lambda} =0, \hspace{4pt}
        -\xi_2-B^*\bar{\lambda} =0
    \end{align}
    Therefore, we have that
    \begin{align}
        0&= \langle \xi_1+A^*\bar{\lambda}, x^k -x^*\rangle+ \langle \xi_2+B^*\bar{\lambda}, y^k-y^*\rangle \nonumber \\
        & =\langle \xi_1, x^k -x^* \rangle +\langle \xi_2, y^k -y^*\rangle +\langle \bar{\lambda}, A(x^k-x^*) \rangle +\langle \bar{\lambda}, B(y^k-y^*) \rangle \nonumber \\
        &\overset{\mathop{(a)}}{\leq} f(x^k) -f(x^*) + g(y^k) -g(y^*) + \langle \bar{\lambda}, Ax^k+By^k-c\rangle
    \end{align}
    where $(a)$ holds because $f$ is convex 
    and $Ax^* + By^* = c$. 
    Therefore, we can get that
    \begin{align}
        -\langle \bar{\lambda}, Ax^k+By^k-c\rangle \leq f(x^k) -f(x^*) + g(y^k) -g(y^*) \label{F-F^*}
    \end{align}
   Besides, using Cauchy-Schwarz inequality, we obtain that
    \begin{align}
        \langle \bar{\lambda}, Ax^k+By^k-c\rangle \leq \|\bar{\lambda}\rVert \cdot\|Ax^k+By^k-c\rVert \label{Cauchy}
    \end{align}
    Summing $\eqref{F-F^*}$ and $\eqref{Cauchy}$, we have that 
    \begin{align}
        0&\leq f(x^k) -f(x^*) + g(y^k) -g(y^*) + \|\bar{\lambda}\rVert \cdot \|Ax^k+By^k-c\rVert \nonumber \\
        & \leq -\frac{\sqrt{k}}{2}\|Ax^k+By^k-c\|^2 +  \frac{2c_1+2c_4}{\sqrt{k+1}} +\frac{c_2}{k+1}+ \frac{c_3}{k(k+1)} +\|\bar{\lambda}\rVert\|Ax^k+By^k-c\|
    \end{align}
    where the last inequality holds because of \eqref{Fk-F*_leq} and $\beta_{k-1} =\sqrt{k}$. 
    Let $\theta_k = \|Ax^k+By^k-c\rVert$, then we obtain that 
    \begin{align}
        -\frac{\sqrt{k}}{2}\theta_k^2 +  \frac{2c_1+2c_4}{\sqrt{k+1}} +\frac{c_2}{k+1}+ \frac{c_3}{k(k+1)} 
        +\|\bar{\lambda}\rVert \theta_k\geq 0
    \end{align}
    Therefore, solving this inequality, we can get that 
    \begin{align}
        \theta_k &\leq \frac{\|\bar{\lambda}\rVert+ \sqrt{\|\bar{\lambda}\rVert^2+2\sqrt{k}\left(\frac{2c_1+2c_4}{\sqrt{k+1}} +\frac{c_2}{k+1}+ \frac{c_3}{k(k+1)} \right)}}{\sqrt{k}} \nonumber \\
        & \leq \frac{\|\bar{\lambda}\rVert}{\sqrt{k}}+\sqrt{\frac{\|\bar{\lambda}\rVert^2}{k}+\frac{4c_1+4c_4}{\sqrt{k(k+1)}} +\frac{2c_2}{(k+1)\sqrt{k}}+ \frac{2c_3}{k\sqrt{k}(k+1)}} \label{Axk-Byk-c}
    \end{align}
    Then, combining $\eqref{Axk-Byk-c}$ with $\eqref{F-F^*}$, we can get 
    \begin{align}
        &f(x^k) -f(x^*) + g(y^k) -g(y^*) \geq -\langle \bar{\lambda}, Ax^k+By^k-c \rangle \geq -\| \bar{\lambda} \rVert \cdot \|Ax^k+ By^k-c\rVert \nonumber \\
        & \geq -\frac{\|\bar{\lambda}\rVert^2}{\sqrt{k}}-\|\bar{\lambda}\rVert\sqrt{\frac{\|\bar{\lambda}\rVert^2}{k}+\frac{4c_1+4c_4}{\sqrt{k(k+1)}} +\frac{2c_2}{(k+1)\sqrt{k}}+ \frac{2c_3}{k\sqrt{k}(k+1)}}
        &\label{Fk-F*}
    \end{align}
    Let
    \begin{align}
        &\tau_k=\frac{c_1 + c_4}{\sqrt{k+1}} +\frac{c_2}{k+1}+ \frac{c_3}{k(k+1)} \nonumber \\
        &\gamma_k = \frac{\|\bar{\lambda}\rVert}{\sqrt{k}}+\sqrt{\frac{\|\bar{\lambda}\rVert^2}{k}+
        \frac{2c_1+ 2c_4}{\sqrt{k(k+1)}} +\frac{2c_2}{(k+1)\sqrt{k}}+ \frac{2c_3}{k\sqrt{k}(k+1)}} \nonumber 
    \end{align}
    Combining $\eqref{Fk-F*_leq_one_size}$ with $\eqref{Fk-F*_leq}$, we get
    \begin{align}
        |f(x^k) -f(x^*) + g(y^k) -g(y^*)| \leq \max\left\{ \tau_k,\|\bar{\lambda}\rVert \gamma_k \right\}
    \end{align}
    Therefore, we get the desired result.
\end{proof}

\subsection{KL property and convergence analysis of the whole sequence}
Now, we have established the convergence rates of the function value $f(x^k)+ g(y^k) -f(x^*)- g(y^*)$ and the 
feasible condition $\|Ax^k + By^k - c\rVert$. However, the sequence $\{x^k,y^k\}$ may not be 
convergent. In order to further study the convergence of the sequence, we will use the KL property and
exponent. 

First, we assume that the functions $x \mapsto f(x) - \bar{\lambda}^TAx$ and $y\mapsto g(y) - \bar{\lambda}^TBy$ 
satisfy the KL property with exponent $\alpha$. Then we have that the function $(x,y) \mapsto f(x)+ g(y) -\bar{\lambda}^T(Ax+By-c)$ 
also has the KL exponent $\alpha$(see Theorem 3.3 in \cite{li2018calculus}). 

Next, we introduce a useful lemma which indicate the relationship of KL exponent between the original function 
and the corresponding Lagrangian function. 
\begin{lemma} \label{lemma_of_Lagrangian_KL_exponent}
    Let $f: \mathbb{R}^n \rightarrow \mathbb{R}$ be a proper, closed and convex function. $A\in \mathbb{R}^{m\times n}$ 
    and $b\in \mathbb{R}^m$ satisfy $b \in A \mathrm{ri} \mathrm{dom} f$. Let $F(x) = f(x) + \delta_{\{b\}}(Ax)$.  
    Let $\bar{\lambda}$ be a Lagrangian multiplier for the following problem
    \begin{align}
        \min\hspace{4pt} &f(x) \nonumber \\
        s.t. \hspace{4pt} & Ax = b \nonumber 
    \end{align}  
    Suppose that $\emptyset \neq \Argmin F$ and $F_{\bar{\lambda}}(x) := f(x) -\bar{\lambda}^TAx$ satisfies the KL property with exponent $\alpha$ 
    at an $\bar{x}\in \Argmin F$ and $\emptyset \neq \mathrm{ri} (\Argmin F_{\bar{\lambda}}) \cap \{x: Ax = b\}$, then 
    $F$ also satisfies KL property at $\bar{x}$ with exponent $\alpha$. 
\end{lemma}
\begin{proof}
    First, because $\bar{\lambda}$ is a Lagrangian multiplier, we have that
    \begin{align}
        F(\bar{x}) =\inf F(x) = f(\bar{x}) = \inf F_{\bar{\lambda}}(x) \leq F_{\bar{\lambda}}(\bar{x}) 
        \leq F(\bar{x})   \nonumber
    \end{align}
    and 
    \begin{align}
        \bar{x} \in \Argmin F =\Argmin F_{\bar{\lambda}} \cap \{x: Ax=b\}   \nonumber
    \end{align}
    Second, since $F_{\bar{\lambda}}$ satisfies KL property with exponent $\alpha$ at $\bar{x}$, we have that there 
    exists $\epsilon>0$,  $r_0>0$ and $\bar{c}>0$ such that for any $x$ satisfying $\|x - \bar{x}\rVert \leq \epsilon$ and 
    $F_{\bar{\lambda}}(\bar{x})<F_{\bar{\lambda}}(x)< F_{\bar{\lambda}}(\bar{x})+ r_0$, it holds that 
    \begin{align}
        \mathrm{dist}(x,\Argmin F_{\bar{\lambda}}) \leq \bar{c}(F_{\bar{\lambda}}(x) - F_{\bar{\lambda}}(\bar{x}))^{1-\alpha} \nonumber
    \end{align}
    Finally, for any $x$ satisfying $\|x - \bar{x}\rVert \leq \epsilon$ and 
    $F(\bar{x})<F(x)< F(\bar{x})+ r_0$,  by using these two results, we obtain that there exists a constant $\kappa$ such that 
    \begin{align}
        &\mathrm{dist} (x, \Argmin F)  = \mathrm{dist} \left(x,\Argmin F_{\bar{\lambda}} \cap \{x: Ax=b\}\right) 
        \overset{\mathop{(a)}}{\leq} \kappa \mathrm{dist}(x, \Argmin F_{\bar{\lambda}}) \nonumber \\
        & \leq \kappa \bar{c} \mathrm{dist}(F_{\bar{\lambda}}(x) - F_{\bar{\lambda}}(\bar{x}))^{1-\alpha} = 
        \kappa \bar{c} \mathrm{dist}(F(x) - F(\bar{x}))^{1-\alpha}
    \end{align}
    where $(a)$ holds because of Corollary 3 of \cite{bauschke1999strong}.
    Hence, the desired result follows. 
\end{proof}

Now, we can deduce the convergence rate of the sequences generated by Algorithm \ref{alg1}. 
\begin{theorem}
    Consider the problem \eqref{problem}. Suppose that $f(x) = f_0(x) + \delta_{\Xi}(x)$ and $g(x) = g_0(x) + \delta_{\Delta}(x)$ where 
    $\Xi$, $\Delta$ are two compact and convex sets and $f_0$ and $g_0$ are two real-valued functions.     
    Let $F(x,y)= f(x)+ g(y)+ \delta_{\mathcal{D}}(x,y)$, where $\mathcal{D} = \{(x,y)| Ax+By=c\}$. Suppose that the function $F_{\bar{\lambda}} = f(x) + g(y)- \bar{\lambda}^T(Ax+By-c)$ 
    satisfies the KL property with exponent $\alpha$ at an $(\bar{x},\bar{y}) \in \Argmin F$ where $\bar{\lambda}$ is the Lagrangian multiplier of problem \eqref{problem}  
    and $\emptyset \neq \Argmin F_{\bar{\lambda}}\cap \mathcal{D}$. It hods that there exist $c>0,\eta>0$ and
    a neighborhood $U$ of $(\bar{x}, \bar{y})$ such that for every $(x,y)\in U$,  
    \begin{align} 
        \mathrm{dist}\left((x,y), \Argmin F\right) \leq c\left(f(x) + g(y)-f(x^*)-g(y^*)+\eta\|Ax+By-c\rVert\right)^{1-\alpha} \nonumber
    \end{align} 
    Furthermore, by Theorem \ref{theorem_convergence_Lipschitz}, we obtain that for any $(x^k,y^k)$ in U, it holds that 
    \begin{align}
        \mathrm{dist}\left((x^k,y^k), \Argmin F\right) \leq  c(\max \{\tau_k, \|\bar{\lambda}\rVert\gamma_k\}+\eta\gamma_k)^{1-\alpha}   \nonumber
    \end{align}
    where $\gamma_k$ and $\tau_k$ are defined in Theorem \ref{theorem_convergence_Lipschitz}. 
    \end{theorem}

    \begin{proof}
        Let $\mathcal{C} = \Xi \times \Delta$. Due to Corollary 3 of \cite{bauschke1999strong}, there exist a $\kappa>0$ such that
        \begin{align} 
            \mathrm{dist}(x,\mathcal{C}\cap \mathcal{D}) \leq \kappa \mathrm{dist}(x,\mathcal{D}) 
            \label{corollary_bauschke}
        \end{align}
    
        Next, since $F_{\bar{\lambda}}$ satisfies the KL property with exponent $\alpha$ at $(\bar{x},\bar{y})$, the 
        function $F$ also satisfies the KL property with exponent $\alpha$ thanks to Lemma \ref{lemma_of_Lagrangian_KL_exponent}. Thus, the following
        result holds due to Lemma \ref{error_bound}.
        \begin{align}
            \mathrm{dist} ((x,y),\Argmin F) \leq \bar{c}(F(x,y) - F(\bar{x},\bar{y}))^{1-\alpha} \label{property_of_error_bound}
        \end{align}
        for some $\bar{c} >0$, $\epsilon>0$, $r_0>0$ and $(x,y)$ satisfying $\mathrm{dist}((x,y),(\bar{x},\bar{y})) \leq \epsilon\leq 1$ and 
        $F(\bar{x},\bar{y})<F(x,y)<F(\bar{x},\bar{y}) + r_0$.
        Besides, since $f_0(x)+ g_0(y)$ is convex and hence it is locally Lipschitz continous. 
        Therefore, there exist $\epsilon_1$ and $L_0$ such that if $\|(x,y) - (\bar{x},\bar{y})\rVert \leq \epsilon_1$, 
        $|f_0(x)+g_0(y) - f_0(\bar{x}) -g_0(\bar{y})| \leq L_0\|(x,y) - (\bar{x},\bar{y})\rVert $. 
        Let $(\tilde{x},\tilde{y})= P_{\mathcal{C}\cap \mathcal{D}}((x,y))$ which denote the projection of $(x,y)$ onto $\mathcal{C}\cap \mathcal{D}$.
        Let $\epsilon_2 = \min\{\epsilon, \epsilon_1\}$. Therefore, if $\mathrm{dist}((x,y),(\bar{x},\bar{y}))\leq \epsilon_2$, we have that 
        \begin{align}
            &\mathrm{dist}((x,y), \Argmin F) \leq \mathrm{dist}(P_{\mathcal{C}\cap\mathcal{D}}((x,y)), \Argmin F)
            + \mathrm{dist}((x,y),\mathcal{C}\cap\mathcal{D}) \nonumber \\
            & \overset{\mathop{(a)}}{\leq} \bar{c}(F(P_{\mathcal{C}\cap \mathcal{D}}(x,y)) - F(\bar{x},\bar{y}))^{1-\alpha} + \kappa\mathrm{dist}((x,y),\mathcal{D}) \nonumber \\
            & =\bar{c}(f_0(\tilde{x})+g_0(\tilde{y})-f_0(\bar{x})-g_0(\bar{y}))^{1-\alpha} + \kappa\mathrm{dist}((x,y),\mathcal{D})  \nonumber \\
            & \overset{\mathop{(b)}}{\leq} \bar{c}(f_0(x)+g_0(y)-f_0(\bar{x})-g_0(\bar{y})+L_0\mathrm{dist}((x,y), \mathcal{C}\cap\mathcal{D})) +\kappa \mathrm{dist}((x,y),\mathcal{D}) \nonumber \\
            & \overset{\mathop{(c)}}{\leq} \bar{c}(f_0(x)+g_0(y)-f_0(\bar{x})-g_0(\bar{y})+L_0\mathrm{dist}((x,y), \mathcal{C}\cap\mathcal{D}))^{1-\alpha}
            + \kappa \mathrm{dist}((x,y),\mathcal{C}\cap \mathcal{D})^{1-\alpha} \nonumber \\
            &\leq c(f_0(x)+g_0(y)-f_0(\bar{x})-g_0(\bar{y})+ \kappa_1\mathrm{dist}((x,y),\mathcal{D}))^{1-\alpha}
            \label{error_bound_f+g}
        \end{align} 
        where $(a)$ holds because \eqref{corollary_bauschke} and \eqref{property_of_error_bound}, $(b)$ holds because $f_0(x)+g_0(y)$ is locally Lipschitz and $\epsilon_2 = \min\{\epsilon, \epsilon_1\}$ 
        $(c)$ holds because $\mathrm{dist}((x,y),\mathcal{C}\cap \mathcal{D}) \leq 1$ and the last inequality holds because 
        $a^{1-\alpha}+ b^{1-\alpha} \leq 2^{\alpha} (a+b)^{1-\alpha}$, $c = 2^{\alpha}\bar{c}$, $\kappa_1 = L_0\kappa+\kappa^{\frac{1}{1-\alpha}}$. 
        Let $a_i^T$, $b_i^T$ be the $i$-th row of the matrices $A$ and $B$ respectively and $c_i$ be the $i$-th element of 
        $c$. Furthermore, we suppose that 
        the vecotr $\tilde{a}_i^T = (a_i^T,b_i^T)$ has at leat one nonzero element.
        Then, by letting $\mathcal{D}_i =\{(x,y)| a_i^Tx+b_i^Ty  =c_i\}$, we have that for some $\kappa_D>0$, 
        it holds that
        \begin{align}
            \mathrm{dist}((x,y), \mathcal{D}) = \mathrm{dist}((x,y),\cap_i\mathcal{D}_i) 
            \leq \kappa_D \max_i \mathrm{dist}((x,y),D_i)  \nonumber
        \end{align}
        where the last inequality holds because of Corollary 3 of \cite{bauschke1999strong}. Let $i_0$ be the 
        index such that $\mathrm{dist}((x,y),D_{i_0}) = \max\limits_i \mathrm{dist}((x,y),D_i)$. Then, we have that
        \begin{align}
            &\mathrm{dist}((x,y),D_{i_0})=\frac{\|\tilde{a}_{i_0}(a_{i_0}^Tx+b_{i_0}^Ty-c_{i_0})\rVert}{a_{i_0}^Ta_{i_0}+b_{i_0}^Tb_i}
            \leq \frac{1}{a_{i_0}^Ta_{i_0}+b_{i_0}^Tb_{i_0}} \|a_{i_0}\rVert \|a_{i_0}^Tx +b_{i_0}^Ty- c_{i_0}\rVert \nonumber \\
            & \leq \frac{1}{a_{i_0}^Ta_{i_0}+b_{i_0}^Tb_{i_0}} \|a_{i_0}\rVert \|Ax+By-c\rVert
        \end{align}
        Let $\eta = \frac{\kappa_D\|a_{i_0}\rVert}{a_{i_0}^Ta_{i_0}+b_{i_0}^Tb_{i_0}}$. 
        Combining this result with \eqref{error_bound_f+g}, we obtain that desired results hold. 
    \end{proof}

\section{Proposed Algorithm for H\"older Continuous Functions}
\begin{assumption} \label{Assumption_holder}
    Suppose that $f$ and $g$ are two convex functions and $f = f_1 + f_2$, $g = g_1+ g_2$.
    Besides, $f_1$, $f_2$, $g_1$, $g_2$ are also convex. Suppose that $f_1$ and $f_2$ have H\"older continuous gradients with constant $M_f$, $M_g$
    and exponents $\mu$, $\nu$, i.e.,
    \begin{align}
        f_1(x+y) &\leq f_1(x) + \left\langle \nabla f_1(x), y \right\rangle +\frac{M_f}{\mu +1}
        \|y\rVert^{\mu +1} \nonumber \\
        g_1(x+y) & \leq g_1(x) + \left\langle \nabla g_1(x), y \right\rangle +\frac{M_g}{\nu+1}
        \|y\rVert^{\nu+1} \label{Holder_property}
    \end{align}
\end{assumption}

The Algorithm \ref{alg2} we proposed for H\"older continuous functions is slight different from Algorithm
\ref{alg1}. In proximal step, we use the stepsize $\frac{1}{M_f}$. However, we consider the proximal
point $v^k$ as the descent direction in Algorithm \ref{alg2} and we get $x^{k+1}$ by the convex combination of $x^k$ and the 
$v^k$.

\begin{algorithm}
	%\textsl{}\setstretch{1.8}
	\caption{ Algorithm for H\"older Continuous Functions}
	\label{alg2}
	\begin{algorithmic}
		\STATE \textbf{Step 0} Choose $x^0 \in \mathrm{dom}f, y^0 \in \mathrm{dom}g$, $r_0 >0$ and the sequences $\{\alpha_k\}$,
        $\{r_k\}$ with $\lim\limits_{k\rightarrow \infty}\alpha_k = 0$, $\lim\limits_{k\rightarrow\infty}r_k=\infty$. Compute $\beta= \lambda_{\max}({A^*A})$.
		\STATE \textbf{Step 1} For $k=0,1,\cdots$ compute
            \begin{align}
                &v^k \in \Argmin \langle \nabla f_1(x^k) + \beta_{k}A^*(Ax^k+By^k-c), x-x^k\rangle +\frac{M_f+\lambda_{A} \beta_{k}}{2}\|x-x^k\rVert^2 +f_2(x) \nonumber \\
                &x^{k+1} = x^k+\alpha_k(v^k - x^k)  \\
                & u^k \in \Argmin \langle\nabla g_1(y^k) + \beta_{k}B^*(Ax^{k+1} +By^k-c),y\rangle +g_2(y) \nonumber \\
                & y^{k+1} = y^{k} +\alpha_k(u^k-y^k)  \label{get_y_k_holder}
            \end{align}
	\end{algorithmic}  
\end{algorithm}

\begin{theorem} \label{Theorem_of_convergence_Holder}
    Consider problem \eqref{problem} and suppose that Assumption \ref{Assumption 1}, \ref{Assumption 2}, \ref{Assumption_holder} and the 
    CQ \eqref{CQ} hold. Recall that $\mu$ and $\nu$ are H\"older exponents of $f_1$ and $f_2$ and we define 
    $\kappa = \min\{\mu,\nu\}$. 
    Suppose that $x^*$, $y^*$ and $\bar{\lambda}$ are defined in lemma \ref{optimal}. 
    We choose that $\beta_n = (n+1)^{\delta}$ for some $\delta\in (0,1]$ and the stepsize $\alpha_n = \frac{2}{n+2}$.  
    Let the sequences $\{x^n\}$, $\{y^n\}$, $\{v^n\}$ and $\{u^n\}$ be generated by 
    Algorithm \ref{alg2}.  
    Then, for all $n\in \mathbb{N}$, we have the following results hold:
    
    $\mathrm{(\romannumeral1)}$ If $\kappa \geq \frac{1}{2}$, we can choose $\delta =\frac{1}{2}$
    and then we have that 
    \begin{align}
        \left|f(x^{n}) + g(y^{n})  - f(x^*) -g(y^*) \right| \leq \frac{c_1+ c_2+ m_0}{\sqrt{n+1}} 
        +\frac{\|\bar{\lambda} \rVert c_0}{\sqrt{n}} \nonumber 
    \end{align}
    and 
    \begin{align}
        \| Ax^n + By^n -c\rVert \leq \frac{c_0}{\sqrt{n}}  \nonumber
    \end{align}

    $\mathrm{(\romannumeral2)}$ If $\kappa < \frac{1}{2}$, we can choose that $\delta \in[\kappa, 1-\kappa]$.
    And then we obtain 
    \begin{align}
        \left| f(x^n) + g(y^n) -f(x^*) -g(y^*)\right| \leq \frac{c_1+ c_2 +m_0+ \|\bar{\lambda}\rVert c_0}{(n+1)^{\kappa}} \nonumber
    \end{align} 
    and
    \begin{align}
        \|Ax^n + By^n - c\rVert \leq \frac{c_0}{\sqrt{n^{\delta}(n+2)^{\kappa}}} \nonumber
    \end{align}
    where
    \begin{align}
        &c_1= (4D_2+2\lambda_{A} D_1D_x+ 2\lambda_{B} D_y^2),\hspace{4pt} c_2=2M_fD_1D_x+2M_gD_y^2,\hspace{4pt}c_0 = \|\bar{\lambda}\rVert +\sqrt{\|\bar{\lambda}\rVert^2+2c_1+2c_2+m_0} \nonumber \\
        &m_0 = 12m_1 + 12m_2,\hspace{4pt} m_1 =\frac{2^{\mu+1}M_fD_x^{\mu+1}}{1-\mu^2},\hspace{4pt} m_2= \frac{2^{\nu+1}M_gD_y^{\nu+1}}{1-\nu^2} \nonumber
    \end{align} 
    Here, $D_x$ and $D_y$ are the diameters of $\mathrm{dom}f$ and $\mathrm{dom}g$, 
    $D_1 = \max\limits_{x^1,x^2\in\mathrm{dom}f}\{\|x^*-x^1\rVert+\|x^*-x^2\rVert\}$, 
    $D_1 = \max\limits_{x\in\mathrm{dom}f,y\in\mathrm{dom}g}\left| \langle Ax,By \rangle \right|$ and $M_f, M_g$ 
    are the H\"older constants of $f_1$ and $g_1$.  
\end{theorem}
\begin{proof}

    Recall that $f_1$ has H\"older continuous gradient with constants $M_f$, $\mu$ and $\lambda_{A} = \lambda_{\max}(A^*A)$. We have that
    \begin{align}
        &f_1(x^{k+1})+ \beta_{k}h(x^{k+1},y^k) +f_2(x^{k+1})  \nonumber \\
        \overset{\mathop{(a)}}{\leq}& f_1(x^k) +\beta_{k}h(x^k,y^k) +\langle \nabla f_1(x^k)+ \beta_{k}A^*(Ax^k+By^k-c), x^{k+1} -x^k\rangle + \frac{M_f}{\mu+1}\|x^{k+1} - x^{k}\rVert^{\mu+1} \nonumber \\
        &+\frac{\lambda_{A} \beta_k}{2}\|x^{k+1}-x^k\rVert^2 +f_2(x^{k+1})  \nonumber \\
        \overset{\mathop{(b)}}{=}& f_1(x^k) +\beta_kh(x^k,y^k) +\alpha_k\langle \nabla f_1(x^k)+\beta_{k}A^*(Ax^k+By^k-c), v^k -x^k\rangle +\frac{M_f}{2}\alpha_k\|v^k-x^k\rVert^2    \nonumber \\
        &-\frac{M_f}{2}\alpha_k\|v^k-x^k\rVert^2+\frac{M_f}{\mu+1}\alpha_k^{\mu +1}\|v^k - x^{k}\rVert^{\mu+1}+\frac{\lambda_{A} \beta_k} {2}\alpha_k^2 \|v^k-x^k\rVert^2+f_2(x^{k+1})  \nonumber \\
        \overset{(c)}{\leq} &f_1(x^k) +\beta_kh(x^k,y^k) +\alpha_k\langle \nabla f_1(x^k)+\beta_{k}A^*(Ax^k+By^k-c), v^k -x^k\rangle +\frac{M_f}{2}\alpha_k\|v^k-x^k\rVert^2    \nonumber \\
        &-\frac{M_f}{2}\alpha_k\|v^k-x^k\rVert^2+\frac{M_f}{\mu+1}\alpha_k^{\mu +1}\|v^k - x^{k}\rVert^{\mu+1}+\frac{\lambda_{A} \beta_k} {2}\alpha_k \|v^k-x^k\rVert^2+f_2(x^{k+1})  \nonumber \\
        \leq& f_1(x^k) +\beta_kh(x^k,y^k) +\alpha_k\langle \nabla f_1(x^k)+ \beta_{k}A^*(Ax^k+By^k-c), v^k -x^k\rangle +\frac{M_f +\lambda_{A} \beta_k}{2}\alpha_k\|v^k-x^k\rVert^2    \nonumber \\
        &-\frac{M_f}{2}\alpha_k\|v^k-x^k\rVert^2+\frac{M_f}{\mu+1}\alpha_k^{\mu +1}D_x^{\mu+1}+f_2(x^{k+1})  \label{f_1_poposition_1_holder}
    \end{align}
    where $(a)$ holds because of \eqref{Holder_property} and 
    $r_kh(\cdot,y^k)$ having Lipschitz continuous gradient with constant $\beta_k\lambda_{A}$, 
    $(b)$ holds because of \eqref{get_y_k_holder} and $(c)$ holds because $\alpha_k^2 \leq \alpha_k$ 
    for every $\alpha_k \in [0,1]$. Besides, the last inequality holds since $\|v^k-x^k\rVert \leq D_x$. 
    By the fact that $f_2$ is convex and the property of the proximal gradient method,  we obtain that
    \begin{align}
        &\alpha_k\langle \nabla f_1(x^k)+\beta_kA^*(Ax^k+By^k-c), v^k -x^k\rangle + \frac{M_f+\lambda_{A} \beta_{k}}{2}\alpha_k\|v^k - x^{k}\rVert^{2} + f_2(x^{k+1}) \nonumber \\
        \leq& \alpha_k (f_1(x^*)-f_1(x^k))+ \alpha_k\beta_k\langle  A^*(Ax^k+By^k-c), x^* -x^k\rangle + \frac{M_f+\lambda_{A} \beta_{k}}{2}\alpha_k\|x^* - x^{k}\rVert^2 + \alpha_k f_2(x^*)  \nonumber \\
        & +(1-\alpha_k)f_2(x^k)-\frac{M_f+\lambda_{A} \beta_{k}}{2}\alpha_k\|x^* - x^{k+1}\rVert^2\label{f_1_poposition_2_holder}
    \end{align}
    Then, combining \eqref{f_1_poposition_1_holder} with \eqref{f_1_poposition_2_holder}, we get the following result.
    \begin{align}
         &f(x^{k+1})+ \beta_{k}h(x^{k+1},y^k) -f(x^*) \nonumber \\
         \leq& (1-\alpha_k)(f(x^k) -f(x^*)) +\beta_{k}h(x^k,y^k)+\alpha_k\langle Ax^k+By^k-c,Ax^*-Ax^k\rangle +\frac{M_f}{\mu+1}\alpha_k^{\mu +1}D_x^{\mu+1} \nonumber \\
         & -\frac{L_f+\lambda_{A} \beta_{k}}{2}\alpha_k\left(\|x^* - x^{k}\rVert^2 -  \|x^* - x^{k+1}\rVert^2\right) -\frac{M_f}{2}\alpha_k\|v^k-x^k\rVert^2 \nonumber \\
         =& (1-\alpha_k)(f(x^k) -f(x^*)) +\beta_{k}h(x^k,y^k)+\alpha_k\langle Ax^k+By^k-c,Ax^*-Ax^k\rangle +\frac{M_f}{\mu+1}\alpha_k^{\mu +1}D_x^{\mu+1} \nonumber \\
         & -\frac{L_f+\lambda_{A} \beta_{k}}{2}\alpha_k\left(\|x^* - x^{k}\rVert -  \|x^* - x^{k+1}\rVert\right)\left(\|x^* - x^{k}\rVert + \|x^* - x^{k+1}\rVert\right) -\frac{M_f}{2}\alpha_k\|v^k-x^k\rVert^2 \nonumber \\
         \overset{\mathop{(a)}}{\leq}& (1-\alpha_k)(f(x^k) -f(x^*)) +\beta_{k}h(x^k,y^k)+\alpha_k\langle Ax^k+By^k-c,Ax^*-Ax^k\rangle +\frac{M_f}{\mu+1}\alpha_k^{\mu +1}D_x^{\mu+1} \nonumber \\
         & +\frac{L_f+\lambda_{A} \beta_{k}}{2}\alpha_k\|x^*  - x^{k+1}\rVert\left(\|x^* - x^{k}\rVert + \|x^* - x^{k+1}\rVert\right) -\frac{M_f}{2}\alpha_k\|v^k-x^k\rVert^2 \nonumber \\
         \overset{\mathop{(b)}}{\leq}& (1-\alpha_k)(f(x^k) -f(x^*)) +\beta_{k}h(x^k,y^k)+\alpha_k\langle Ax^k+By^k-c,Ax^*-Ax^k\rangle  \nonumber \\
         &+\frac{M_f+\lambda_{A} \beta_k}{2}\alpha_k^2D_1D_x + \frac{M_f}{\mu+1}\alpha_k^{\mu +1}D_x^{\mu+1}
         \label{f_k+1_holder}
    \end{align}
    where $(a)$ holds because $-\left(\|x^* - x^k\rVert - \|x^* - x^{k+1}\rVert\right) \leq \|x^k-x^{k+1}\rVert$ 
    and $(b)$ holds since  $\|v^k -x^k\rVert \leq D_x$ and $ \|x^* - x^k\rVert +\|x^* -x^{k+1}\rVert \leq D_1$ 
    where $D_1 = \max\limits_{x_1,x_2\in\mathrm{dom}f}\|x^* - x^k\rVert+ \|x^*-x^{k+1}\rVert$ which is 
    defined in our statement of theorem \ref{Theorem_of_convergence_Holder}. 

    Next, by the same methods in \eqref{g_k+1} and \eqref{F_k-F_k+1}, we have that 
    \begin{align}
         &g(y^{k+1}) +\beta_{k}h(x^{k+1},y^{k+1}) -g(y^*)  \nonumber \\
        \leq& (1-\alpha_k)(g(y^k)-g(y^*))+\beta_{k}h(x^{k+1},y^k)+ \alpha_k\beta_{k}\langle Ax^{k+1} +By^k -c, By^* -By^k\rangle  \nonumber \\
        &+\frac{M_g+\lambda_{B} \beta_{k}}{2} \alpha_k^2 D_y^2 +\frac{M_g}{\nu+1}\alpha_k^{\nu +1}D_y^{\nu+1} \label{g_k+1_holder}
    \end{align}
    and
    \begin{align}
        &f(x^{k+1}) +g(y^{k+1}) +\beta_{k}h(x^{k+1},y^{k+1}) -f(x^*) -g(y^*) \nonumber \\
        \leq& (1-\alpha_k)\left(f(x^k) +g(y^k)+\beta_{k-1}h(x^k,y^k) -f(x^*)-g(y^*)\right)+\alpha_k\beta_{k}\langle Ax^k+By^k-c,Ax^*-Ax^k\rangle \nonumber \\
        &+ \alpha_k\beta_{k}\langle Ax^{k+1} +By^k -c, By^* -By^k\rangle +\frac{M_f+\lambda_{A} \beta_{k}}{2}\alpha_k^2D_1D_x+ \frac{M_g +\lambda_{B} \beta_{k}}{2}\alpha_k^2D_y^2
        \nonumber \\
        & +(1-\alpha_k)(\beta_{k}-\beta_{k-1})h(x^k,y^k) + \alpha_k\beta_{k} h(x^k,y^k) +\frac{M_f}{\mu+1}\alpha_k^{\mu +1}D_x^{\mu+1} +\frac{M_g}{\nu+1}\alpha_k^{\nu +1}D_y^{\nu+1} \label{F_k-F_k+1_holder}
    \end{align}
    Similarly, also by the lemma \ref{lemma_of_rk} and recalling the definition of $D_2$ which is defined in theorem \ref{theorem_convergence_Lipschitz}, 
    we obtain that 
    \begin{align}
        &(1-\alpha_k)(\beta_{k}-\beta_{k-1})h(x^k,y^k) + \alpha_k \beta_{k}h(x^k,y^k) + \alpha_k\beta_{k}\langle Ax^k+By^k-c,Ax^*-Ax^k\rangle \nonumber\\
        &+ \alpha_k\beta_{k}\langle Ax^{k+1} +By^k -c, By^* -By^k\rangle \nonumber \\
        \overset{\mathop{(a)}}{\leq}& \alpha_k\beta_{k} \langle Ax^k -Ax^{k+1}, By^k-By^*\rangle \nonumber  
        =\alpha_k^2\beta_k \langle Av^k -Ax^k,By^k-By^* \rangle  \nonumber \\
        =& \alpha_k\beta_k \left(\langle Av^k,By^k \rangle-\langle Av^k, By^*\rangle - 
        \langle Ax^{k},By^k \rangle+ \langle Ax^{k}, By^* \rangle\right)  \nonumber \\
        \leq& \alpha_k^2\beta_k \left(\left|\langle Av^k,By^k \rangle \right|+\left|\langle Av^k, By^*\rangle\right| + 
        \left|\langle Ax^{k},By^k \rangle\right|+ \left|\langle Ax^{k}, By^* \rangle\right|\right) 
        \leq 4\alpha_k^2\beta_k D_2 \label{F_k-F_k+1_2_holder}
    \end{align} 
    where $(a)$ holds because of \eqref{F_k-F_k+1_2}.
    Combining $\eqref{F_k-F_k+1_holder}$ with $\eqref{F_k-F_k+1_2_holder}$, we have that
    \begin{align}
        &f(x^{k+1}) +g(y^{k+1})+ \beta_{k}h(x^{k+1},y^{k+1}) -f(x^*) -g(y^*) \nonumber \\
        \leq& (1-\alpha_k)\left(f(x^k) +g(y^k)+\beta_{k-1}h(x^k,y^k) -f(x^*)-g(y^*)\right)+ \alpha_k^2\beta_{k}D_2   \nonumber \\
        &+ \frac{M_f+\lambda_{A} \beta_{k}}{2}\alpha_k^2D_1D_x +\frac{M_g +\lambda_{B} \beta_{k}}{2}\alpha_k^2D_y^2+\frac{M_f}{\mu+1}\alpha_k^{\mu +1}D_x^{\mu+1} +\frac{M_g}{\nu+1}\alpha_k^{\nu +1}D_y^{\nu+1}
    \end{align}
    Let $\Gamma_k =k(k+1) \left(f(x^k) +g(y^k)+\beta_{k-1}h(x^k,y^k) -f(x^*)-g(y^*)\right)$ and multiply $(k+1)(k+2)$ to both sides of the above inequality.
    Then, we have that 
    \begin{align}
        &\Gamma_{k+1} \nonumber \\ 
        \leq& \Gamma_k +4(k+1)(k+2)\alpha_k^2\beta_kD_2 + \frac{M_f+\lambda_{A} \beta_k}{2}(k+1)(k+2)\alpha_k^2D_1D_x+ \frac{M_g +\lambda_{B} \beta_k}{2}(k+1)(k+2)\alpha_k^2D_y^2 \nonumber \\
        &+\frac{M_f}{\mu+1}\alpha_k^{\mu +1}(k+1)(k+2)D_x^{\mu+1} +\frac{M_g}{\nu+1}\alpha_k^{\nu +1}(k+1)(k+2)D_y^{\nu+1} \nonumber \\
        \overset{\mathop(a)}{\leq} &\Gamma_k +16D_2 \beta_k + 2(M_f+\lambda_{A} \beta_k)D_1D_x+ 2(M_g +\lambda_{B} \beta_k)D_y^2 \nonumber\\ 
        &+\frac{2^{\mu+1}}{\mu+1}M_f(k+2)^{1-\mu}D_x^{\mu+1}+\frac{2^{\nu+1}}{\nu+1}M_g(k+2)^{1-\nu}D_y^{\nu+1}\nonumber \\
         \leq& \Gamma_k + (16D_2+2\beta D_1D_x+ 2\rho D_y^2)\beta_k +\frac{2^{\mu+1}}{\mu+1}M_f(k+2)^{1-\mu}D_x^{\mu+1} \nonumber\\
        &+\frac{2^{\nu+1}}{\nu+1}M_g(k+2)^{1-\nu}D_y^{\nu+1}+2M_fD_1D_x+2M_gD_y^2 
    \end{align}
    where $(a)$ holds because $(k+1)(k+2) \alpha_k^2 =(k+1)(k+2) \frac{4}{(k+2)^2} \leq 4$ and $(k+1)(k+2) \alpha_k^{\mu+1} \leq2^{\mu+1}(k+2)^{1-\mu}$, we obtain that 
    \begin{align}
        \Gamma_{k+1} \leq& \Gamma_k +16D_2 \beta_k + 2(M_f+\lambda_{A} \beta_k)D_1D_x+ 2(M_g +\lambda_{B} \beta_k)D_y^2 \nonumber\\ 
        &+\frac{2^{\mu+1}}{\mu+1}M_f(k+2)^{1-\mu}D_x^{\mu+1}+\frac{2^{\nu+1}}{\nu+1}M_g(k+2)^{1-\nu}D_y^{\nu+1}\nonumber \\
         \leq& \Gamma_k + (16D_2+2\lambda_{A} D_1D_x+ 2\lambda_{B} D_y^2)\beta_k +\frac{2^{\mu+1}}{\mu+1}M_f(k+2)^{1-\mu}D_x^{\mu+1} \nonumber\\
        &+\frac{2^{\nu+1}}{\nu+1}M_g(k+2)^{1-\nu}D_y^{\nu+1}+2M_fD_1D_x+2M_gD_y^2 
    \end{align}
    Summing the both sides of the above inequality, we obtain that the following statement holds .
    \begin{align}
        \Gamma_n &\leq \sum_{k = 0}^{n-1} \left[ (16D_2+2\lambda_{A} D_1D_x+ 2\lambda_{B} D_y^2)\beta_k +\frac{2^{\mu+1}}{1+\mu}M_fD_x^{\mu+1}(k+2)^{1-\mu }\right. \nonumber \\
        &\left. +\frac{2^{\nu+1}}{1+\nu}D_y^{\nu+1}M_g(k+2)^{1-\nu}+ (2M_fD_1D_x+2M_gD_y^2 ) \right]  \nonumber \\
         \overset{\mathop{(a)}}{\leq}& \beta_{n}n(16D_2+2\lambda_{A} D_1D_x+ 2\lambda_{B} D_y^2) +(2M_fD_1D_x+2M_gD_y^2 )n  \nonumber \\
        &+\frac{2^{\mu+1}}{1+\mu}M_fD_x^{\mu+1}\sum_{k=0}^{n-1}(k+2)^{1-\mu} + \frac{2^{\nu+1}}{1+\nu}M_fD_x^{\nu+1}\sum_{k=0}^{n-1}(k+2)^{1-\nu} \nonumber \\
         \overset{\mathop{(b)}}{\leq}& \beta_{n}n(16D_2+2\lambda_{A} D_1D_x+ 2\lambda_{B} D_y^2)+ \frac{2^{\mu+1}M_fD_x^{\mu+1}}{1-\mu^2}(n+2)^{2-\mu} \nonumber \\
        &+ \frac{2^{\nu+1}M_gD_y^{\nu+1}}{1-\nu^2}(n+2)^{2-\nu}+(2M_fD_1D_x+2M_gD_y^2 )n 
    \end{align}
    where $(a)$ holds because $\beta_{k} \leq \beta_{n-1}$ for every $ 0 \leq k \leq n-1$, $(b)$ holds because the function
    $x \mapsto x^{1-\delta}$ is increasing when $\delta \in (0,1]$. 
    Therefore, we have that
    \begin{align}
        &f(x^n) +g(y^n)+\beta_{n}h(x^n,y^n) -f(x^*)-g(y^*) \leq  \frac{\beta_{n}c_1}{n+1} +\frac{m_1(n+2)^{2-\mu}}{n(n+1)}+\frac{m_2(n+2)^{2-\nu}}{n(n+1)}+ \frac{c_2}{n+1} \nonumber \\
        &= \frac{\beta_{n}c_1}{n+1} +\frac{n+2}{n+1}\cdot\frac{n+2}{n}\cdot\frac{m_1}{(n+2)^{\mu}}+\frac{n+2}{n+1}\cdot\frac{n+2}{n}\cdot\frac{m_2}{(n+2)^{\nu}}+ \frac{c_2}{n+1} \nonumber\\
        &\leq  \frac{\beta_{n}c_1}{n+1} +\frac{6m_1}{(n+2)^{\mu}}+\frac{6m_2}{(n+2)^{\nu}}+ \frac{c_2}{n+1}  
        \label{Fk-F*_leq_holder}
    \end{align}
    where $c_1= (16D_2+2\lambda_{A} D_1D_x+ 2\lambda_{B} D_y^2)$, $c_2=2M_fD_1D_x+2M_gD_y^2$, $m_1 =\frac{2^{\mu+1}M_fD_x^{\mu+1}}{1-\mu^2} $ and $m_2= \frac{2^{\nu+1}M_gD_y^{\nu+1}}{1-\nu^2}$. 
    By the same method in \eqref{Axk-Byk-c}, we obtain that
    \begin{align}
        \|Ax^n+By^n-c\rVert &\leq
        \frac{\|\bar{\lambda}\rVert + \sqrt{\|\bar{\lambda}\rVert^2+2\beta_n\left(\frac{c_1\beta_n+c_2}{n+1}+\frac{6m_1}{(n+2)^{\mu}}+\frac{6m_2}{(n+2)^{\nu}}\right)}}{\beta_k}  \nonumber \\
        & \leq \frac{\|\bar{\lambda}\rVert}{\beta_n}+\sqrt{\frac{\|\bar{\lambda}\rVert^2}{\beta_n^2}+\frac{2c_1}{n+1}+\frac{2c_2}{\beta_n(n+1)}+\frac{12m_1}{\beta_n(n+2)^{\mu}}+\frac{12m_2}{\beta_n(n+2)^{\nu}}} \nonumber 
    \end{align}
    Recall that $\kappa = \min\{\mu,\nu\}$ and $\tau_n = \min\left\{{\beta_n}, \sqrt{n+1},\sqrt{\beta_n(n+2)^{\kappa}}\right\}$. Then, we have that
    \begin{align}
         \|Ax^n+By^n-c\rVert & \leq \frac{c_0}{\tau_n} \label{Axk-Byk-c_holder}
    \end{align}
    where $c_0 = \|\bar{\lambda}\rVert +\sqrt{\|\bar{\lambda}\rVert^2+2c_1+2c_2+12m_1+12m_2}$. 
    Then, similar to \eqref{Fk-F*}, we can  get that 
    \begin{align}
        f(x^n) -f(x^*) + g(y^n) -g(y^*) \geq -\frac{\|\bar{\lambda}\rVert c_0}{\tau_{n}} \label{Fk-F*_holder}
    \end{align}
   and
    \begin{align}
        |f(x^n) -f(x^*) + g(y^n) -g(y^*)| \leq \max\left\{\frac{\beta_nc_1}{n+1} + \frac{c_2}{n+1}  +\frac{m_0}{(n+2)^{\kappa}}, \frac{\|\bar{\lambda}\rVert c_0}{\tau_n}\right\}
    \end{align}
    where $m_0 = 12m_1+12m_2$. Then we consider two cases with respect to $\varepsilon$.
    If $\kappa \geq \frac{1}{2}$, recall that $\beta_n = n^{\delta}$, $\delta\in (0,1]$. To get 
    the fastest convergence rate, we choose 
    $\delta = \frac{1}{2}$, then we have that
    \begin{align}
        \tau_n = \min\left\{ n^{\delta}, \sqrt{n+1}, \sqrt{n^{\delta}(n+2)^{\kappa}} \right\} =\sqrt{n} \nonumber
    \end{align}
    and
    \begin{align}
        \frac{\beta_nc_1}{n+1} + \frac{c_2}{n+1}  +\frac{m_0}{(n+2)^{\kappa}} \leq \frac{c_1+c_2+m_0}{\sqrt{n+1}} \nonumber   
    \end{align}
    for large $k$.
    Therefore, we have that
    \begin{align}
        \left|f(x^n) -f(x^*) + g(y^n) -g(y^*)\right| \leq \max\left\{\frac{c_1+c_2+m_0}{\sqrt{n+1}}, \frac{\|\bar{\lambda}\rVert c_0}{\sqrt{n}} \right\} 
        \leq \frac{c_1+c_2+m_0}{\sqrt{n+1}} + \frac{\|\bar{\lambda}\rVert c_0}{\sqrt{n}}
    \end{align}
    and
    \begin{align}
        \|Ax^n +By^n - c\rVert \leq \frac{c_0}{\sqrt{n}}
    \end{align}
    
    If $\kappa < \frac{1}{2}$, we can choose $\delta \in [\kappa, 1-\kappa] $. Then we have that
    \begin{align}
        \frac{\beta_nc_1}{n+1} + \frac{c_2}{n+1}  +\frac{m_0}{(n+2)^{\kappa}} \leq \frac{c_1+c_2+m_0}{(n+1)^{\kappa}}
    \end{align}
    and
    \begin{align}
        \tau_n = \min\left\{ n^{\delta}, \sqrt{n+1}, \sqrt{n^{\delta}(n+2)^{\kappa}} \right\} 
        =\sqrt{n^{\delta}(n+2)^{\kappa}} \nonumber
    \end{align}
    Therefore, we have that
    \begin{align}
        \left|f(x^n) -f(x^*) + g(y^n) -g(y^*)\right| \leq \max\left\{\frac{c_1+c_2+m_0}{(n+1)^{\kappa}}, 
        \frac{\|\bar{\lambda}\rVert c_0}{\sqrt{n^{\delta}(n+2)^{\kappa}}} \right\} \leq 
        \frac{c_1+c_2+m_0+\|\bar{\lambda}\rVert c_0}{(n+1)^{\kappa}}
    \end{align}
    and
    \begin{align}
        \|Ax^n +By^n - c\rVert \leq \frac{c_0}{\sqrt{n^{\delta}(n+2)^{\kappa}}}
    \end{align}
\end{proof}




\section{Examples}
\subsection{Hankel Matrix Rank Minimization}

Consider the following problem
\begin{align} \label{Hankel Matrix}
    \min\limits \hspace{4pt}&\frac{1}{2}\|\mathcal{A}(y)-b\rVert^2 \nonumber\\
    s.t. \hspace{4pt} &\| \mathcal{H}(y) \rVert_* \leq \sigma   
\end{align}
where $\mathcal{A}: \mathbb{R}^{m\times n(k+j-1)} \rightarrow \mathbb{R}^p$ is a linear operator,   
$y =\left(y_0, y_1,...,y_{j+k-2}\right)$ is an $m \times n(k+j-1)$ matrix and each
$x_i$ is an $m\times n$ matrix. In this paper, $\mathcal{A}(y)$ is the operator that misses some elements
$y_i$ in $y$ and then stretches it into a vector.  Therefore, we can obtain that
\begin{align}
    \lambda_{\max}(\mathcal{A}^*\mathcal{A}) \leq 1 \nonumber
\end{align}
Besides, $\mathcal{H}(y) := H_{m, n, j, k}(y)U$, where
\begin{align}
    H_{m,n,j,k}(y) = \left(
        \begin{array}{cccc}
            y_0 & y_1 & \cdots &y_{k-1} \\
            y_1 & y_2 & \cdots &y_{k} \\
            \vdots & \vdots &  &\vdots \\
            y_{j-1} &y_{j}  &  &y_{j+k-2} 
        \end{array} \nonumber
    \right) \in \mathbb{R}^{mj\times nk}
\end{align}
and
$U$ is a matrix whose columns are orthogonal. Therefore, We can also obtain that 
\begin{align}
    \| \mathcal{H}^*\mathcal{H}(y) \rVert \leq \mathbf{r}\| y\rVert \nonumber
\end{align}
where $\mathbf{r} = \min\{ j,k\}$.

Because $\|\mathcal{H}(x)\rVert_* \leq \sigma$, we have that
\begin{align}
    tr\left(\mathcal{H}(x)^T\mathcal{H}(x)\right) \leq \sigma^2 \nonumber
\end{align}
Notice that 
\begin{align}
    &tr\left(\mathcal{H}(x)^T\mathcal{H}(x)\right)  = 
    tr\left(H_{m,n,j,k}^T H_{m,n,j,k}\right) =\sum\limits_{i =0}^{k-1}\sum\limits_{l=i}^{l+j-1} 
    tr\left(x_l^Tx_l\right) \geq \sum\limits_{i=0}^{k+j-2} tr\left(x_i^Tx_i\right)
    = tr\left(x^Tx\right) \nonumber
\end{align}
Therefore, we have that $\|x\rVert_F \leq \sigma$.
Then the problem \eqref{Hankel Matrix} can be written as 
\begin{align}\label{Hankel_Matrix_reformulation}
    \min\hspace{4pt} f(x) + g(y) +\frac{r_k}{2}\|y-\mathcal{H}(x) \rVert_F^2 
\end{align}
where $f(x) = \frac{1}{2}\|\mathcal{A}(x)-b\rVert^2 + \delta_{\|\cdot\rVert_F \leq \sigma}(x)$, 
$g(y) = \delta_{\|\cdot\rVert_* \leq \sigma}(y), r_k = \sqrt{k}$, $\alpha_k =\frac{2}{k+2}$.
Therefore, we can write the closed form of algorithm $\ref{alg1}$.
First, we can get SVD decomposition of $y^k - \mathcal{H}(y^k)$, i.e.
\begin{align}
    y^k -\mathcal{H}(x^k) = U\Sigma V^T  = U \left(\begin{array}{cccc}
        \sigma_1 &  & & \\
          &\sigma_2 & & \\
          &   & \ddots & \\
          &   &     &\sigma_s
    \end{array}\right)  V^T \nonumber
\end{align}
Suppose that $i_0$ is the index of the maximum element of $\Sigma$, $U_{i_0}$ is the $i_0$-th row of $U$ and 
$V^T_{i_0}$ is the $i_0$-th column of $V^T$. 
In the Frank-Wolfe step, 
we need  to solve the following subproblem
\begin{align}
    \min\limits_{\|y\rVert_* \leq \sigma} \hspace{4pt} \langle y^k - \mathcal{H}(x^k), y\rangle \nonumber
\end{align}
whose solution is $\sigma_{i_0}U_{i_0}V_{i_0}^T$. Therefore, the algorithm $\ref{alg1}$ for solving 
problem $\eqref{Hankel_Matrix_reformulation}$ can be written as follows. 
\begin{align}
    \left\{
        \begin{aligned}
            x^{k+1} &= \mathrm{Prox}_{\delta_{\|\cdot\rVert_F \leq \sigma}}\left(x^k-\frac{1}{1+\sqrt{k}\mathbf{r}}\left(\mathcal{A^*}\left(\mathcal{A}(x^k)-b\right)
            +\mathcal{H^*}\left(\mathcal{H}(x^k) -y^k\right)\right)\right) \\
            u^k &= \sigma_{i_0}U_{i_0}V_{i_0}^T \\
            y^{k+1} &= y^k+ \alpha_k(u^k-y^k)
        \end{aligned}
     \right.
\end{align}


\bibliography{sample}


\end{document}